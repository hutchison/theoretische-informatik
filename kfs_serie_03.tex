\begin{enumerate}
  \item
    \begin{enumerate}[(a)]
      \item In der Übung wurde gezeigt, dass sich \DreiSAT{} auf das
        Halteproblem reduzieren lässt.

        Um zu zeigen, dass das Halteproblem wenigstens so schwer wie
        \COLORDrei{} ist, überführen wir \COLORDrei{} in \DreiSAT{}.

        Gegeben sei ein Graph $(V, E)$ mit $|V| = n$ und eine Menge mit drei
        Farben, die wir mit $\{1,2,3\}$ abstrahieren. Wir führen für jeden
        Knoten und jede Farbe ein Literal $v_{i, c}$ (mit $i \in \{1, \ldots,
        n\}$ und $c \in \{1,2,3\}$) ein, welches angibt, ob der Knoten mit
        dieser Farbe gefärbt wurde.

        Die aussagenlogische Formel für \DreiSAT{} ergibt sich aus zwei
        Teilformeln: $K \wedge N$

        Dabei gilt:
        \begin{itemize}
          \item $K$ sind die Bedingungen für die Färbungen der einzelnen Knoten,
          \item $N$ sind die Bedingungen für die unterschiedlichen Färbungen
            benachbarter Knoten
        \end{itemize}

        Ein einzelner Knoten darf nur mit einer Farbe gefärbt werden, also gilt:
        \begin{align*}
          K = \bigwedge\limits_{i \in \{1, \ldots, n\}}
            (v_{i, 1} \vee v_{i, 2} \vee v_{i, 3}) \wedge
            (\neg v_{i, 1} \vee \neg v_{i, 2}) \wedge
            (\neg v_{i, 1} \vee \neg v_{i, 3}) \wedge
            (\neg v_{i, 2} \vee \neg v_{i, 3})
        \end{align*}

        Weiterhin dürfen benachbarte Knoten nicht mit derselben Farbe gefärbt
        werden. Es gilt also für alle $i \in \{1, \ldots, n\}$ und
        $v_j \in N(v_i)$:
        \begin{align*}
          N = \bigwedge\limits
            (\neg v_{i, 1} \vee \neg v_{j, 1}) \wedge
            (\neg v_{i, 2} \vee \neg v_{j, 2}) \wedge
            (\neg v_{i, 3} \vee \neg v_{j, 3})
        \end{align*}

        Alle Klauseln enthalten höchstens drei Literale und die Formel $K \wedge
        N$ befindet sich in konjunktiver Normalform.

      \item Während das Halteproblem nicht entscheidbar ist, weil die
        Turingmaschine unendlich lange laufen würde, ist \COLORDrei{}
        entscheidbar, da man „nur“ endlich viele Möglichkeiten ausprobieren
        muss.
    \end{enumerate}

  \item ---

  \item
    \begin{enumerate}[(a)]
      \item Um zu zeigen, dass $\NELP{} \in NP$ ist, müssen wir zeigen, dass
        eine nichtdeterministische Turingmaschine existiert, die das Problem in
        polynomieller Zeit entscheidet.

        Angelehnt an die Vorlesung agiert die NTM nach folgendem
        Pseudoalgorithmus:
        \begin{enumerate}
          \item „rate“ ein $x \in {\{0, 1\}}^n$
          \item überprüfe $f(x_1, \ldots, x_n) \geq k$ und $Ax \leq b$
        \end{enumerate}
        Da der letzte Schritt in polynomieller Zeit ausgeführt werden kann, gilt
        die Behauptung.

      \item Gegeben sei das Problem \CLIQUE{} über dem Graphen $G = (V, E)$ mit
        $n = |V|$ und $k \in \N$. Für die Reduktion auf \NELP{} betrachten wir
        den Komplementgraph $\overline{G} = (V, V \times V \setminus E)$.

        Die Matrix $A$ ergibt sich aus der Inzidenzmatrix von $\overline{G}$:
        \begin{itemize}
          \item $A$ enthält $n$ Spalten
          \item jede Zeile von $A$ bildet eine Kante aus $\overline{G}$ ab
          \item jede Zelle von $A$ enthält eine $1$, falls der Knoten der
            jeweiligen Spalte auf der Kante liegt; ansonsten eine $0$
        \end{itemize}
        $A$ hat damit $m = \binom{n}{2} - |E|$ Zeilen und $n$ Spalten. Das $k$
        aus \CLIQUE{} ist gleich dem $k$ aus \NELP{}. $c$ ist der
        $n$-dimensionale $1$-Vektor und $b$ ist der $m$-dimensionale $1$-Vektor.

        Der Lösungsvektor $x$ gibt dann an, welche Knoten in der Clique
        enthalten sind.

      \item Die Erzeugung des Komplementgraphen benötigt maximal $\binom{n}{2} =
        \frac{n(n-1)}{2}$ Schritte und um die Matrix $A$ zu erzeugen, benötigen
        wir $n \cdot m = n \cdot \left(\frac{n(n-1)}{2} - |E|\right)$ Schritte.
        Insgesamt liegt der Aufwand in $O(n^3)$ und ist damit polynomiell.

      \item Aus
        \begin{align*}
          G = (\{1,2,3,4,5,6\}, \{12, 13, 15, 23, 25, 34, 35, 45\})
        \end{align*}
        folgt
        \begin{align*}
          \overline{G} = (\{1,2,3,4,5,6\}, \{14, 16, 24, 26, 36, 46, 56\})
        \end{align*}
        Die Matrix $A$ ist demnach
        \begin{align*}
          A & = \begin{pmatrix}
            1 & 0 & 0 & 1 & 0 & 0\\
            1 & 0 & 0 & 0 & 0 & 1\\
            0 & 1 & 0 & 1 & 0 & 0\\
            0 & 1 & 0 & 0 & 0 & 1\\
            0 & 0 & 1 & 0 & 0 & 1\\
            0 & 0 & 0 & 1 & 0 & 1\\
            0 & 0 & 0 & 0 & 1 & 1\\
          \end{pmatrix}
        \end{align*}

        Die maximale Clique hat die Größe $k = 4$ und enthält die Knoten
        $\{1,2,3,5\}$. In \NELP{} ergibt das folgende Ergebnisse:
        \begin{align*}
          Ax & = \begin{pmatrix}
            1 & 0 & 0 & 1 & 0 & 0\\
            1 & 0 & 0 & 0 & 0 & 1\\
            0 & 1 & 0 & 1 & 0 & 0\\
            0 & 1 & 0 & 0 & 0 & 1\\
            0 & 0 & 1 & 0 & 0 & 1\\
            0 & 0 & 0 & 1 & 0 & 1\\
            0 & 0 & 0 & 0 & 1 & 1\\
          \end{pmatrix}
          \begin{pmatrix}
            1\\
            1\\
            1\\
            0\\
            1\\
            0
          \end{pmatrix}
          =
          \begin{pmatrix}
            1\\
            1\\
            1\\
            1\\
            1\\
            0\\
            1
          \end{pmatrix}
          \leq
          \begin{pmatrix}
            1\\
            1\\
            1\\
            1\\
            1\\
            1\\
            1
          \end{pmatrix}
          = b
        \end{align*}
        Der optimale Zielfunktionswert ist
        \begin{align*}
          \sum\limits_{i=1}^n c_i x_i = \sum\limits_{i=1}^n x_i = 4
        \end{align*}
    \end{enumerate}
\end{enumerate}
