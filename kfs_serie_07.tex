\begin{enumerate}[1.]
  \item
    \begin{enumerate}[(a)]
      \item Wir wählen $n = 1$. Damit gilt $x = x_0 R x_1 = y$, also sind alle
        Tupel $(x,y)$ aus $R$ auch in $R^+$ ($R \subseteq R^+$).

      \item Wir starten mit $n = 2$:
        \begin{itemize}
          \item $x = aRbRc = y \Rightarrow (a,c) \in R^+$
          \item $x = aRbRe = y \Rightarrow (a,e) \in R^+$
          \item $x = bReRd = y \Rightarrow (b,d) \in R^+$
        \end{itemize}

        Weiter geht’s mit $n = 3$:
        \begin{itemize}
          \item $x = aRbReRd = y \Rightarrow (a,d) \in R^+$
        \end{itemize}

        Und das war’s auch schon. Damit ist
        $R^+ = \{(a,b), (b,c), (b,e), (e,d), (a,c), (a,e), (b,d), (a,d)\}$

      \item $R^* = R^+ \cup \{(a,a), (b,b), (c,c), (d,d), (e,e)\}$
    \end{enumerate}

  \item
    \begin{enumerate}[(a)]
      \item Damit $E \subseteq R$ gilt, müssen wir für alle $(u,v) \in E$ je
        ein $w_1$ und $w_2$ finden, sodass $w = w_1 u w_2$ und $\tilde{w} = w_1
        v w_2$ und damit $w R \tilde{w}$ gilt.

        Dafür wählen wir $w_1 = w_2 = \epsilon$ und damit ist $w = u$ und
        $\tilde{w} = v$. Da $(u,v) \in E$, gilt $wR\tilde{w}$.

      \item\
        \begin{center}
          \begin{tabular}{llllll}
            \toprule
            $w$ & $\tilde{w}$ & $w_1$ & $u$ & $v$& $w_2$\\
            \midrule
            $aX$ & $acZ$ & $a$ & $X$ & $cZ$ & $\epsilon$\\
            $acZ$ & $aca$ & $ac$ & $Z$ & $a$ & $\epsilon$\\
            $bY$ & $bdU$ & $b$ & $Y$ & $dU$ & $\epsilon$\\
            $bdU$ & $bdb$ & $bd$ & $U$ & $b$ & $\epsilon$\\
            $cZ$ & $ca$ & $c$ & $Z$ & $a$ & $\epsilon$\\
            $dU$ & $db$ & $d$ & $U$ & $b$ & $\epsilon$
          \end{tabular}
        \end{center}

      \item $\{ (S, w) \in \{ S \} \times {(X \cup H)}^* \mid S R^+ w\}
        = \{ (S, aX), (S, acZ), (S, aca), (S, bY), (S, bdU), (S, bdb) \}$

      \item $\{ (S, w) \in \{ S \} \times {(X \cup H)}^* \mid S R^* w\}
        = \{ (S, w) \in \{ S \} \times {(X \cup H)}^* \mid S R^+ w\}
        \cup \{ (S,S) \}$
      \item $\{ aca, bdb \}$
    \end{enumerate}

  \item
    \begin{enumerate}[(a)]
      \item Sei $G = (X, H, E, S)$ mit
        \begin{align*}
          X & = \{a, b, c\}\\
          H & = \{ S \}\\
          E & = \{
            (S, \epsilon),
          (S, a), (S, b), (S, c),
          (S, aSa), (S, bSb), (S, cSc)
          \}
        \end{align*}

      \item Ich nehme mal an, dass wir folgende Aussage beweisen
        sollen:\footnote{Sonst wäre diese Aufgabe die gleiche wie 3 (c).}
        \begin{center}
          Wenn $w \in {\{a, b, c\}}^*$ ein Palindrom ist, dann ist $w \in L_G$.
        \end{center}
        Sei $\ell: X^* \rightarrow \N$ die Wortlängenfunktion.

        Induktionsanfang: sei $w \in {\{a, b, c\}}^*$ und $\ell(w) \leq 1$.
        Ausgehend vom Startsymbol $S$ wurde dann eine der folgenden
        Ableitungsregeln angewandt: $(S, \varepsilon), (S, a), (S, b), (S, c)$.
        Damit ist $w \in L_G$.

        Induktionsvoraussetzung: sei $w \in {\{a, b, c\}}^*$ und $\ell(w) = n$,
        dann ist $w \in L_G$.

        Induktionsschritt: sei $v = v_1 v_2 \ldots v_{n+2} \in {\{a, b, c\}}^*$
        ein Palindrom mit $\ell(v) = n+2$. Dann gilt nach Definition von
        Palindromen:
        \begin{itemize}
          \item Es existiert ein Palindrom $w \in {\{a, b, c\}}^*$ mit $w = v_2
            \ldots v_{n+1}$ und $\ell(w) = n$, welches nach
            Induktionsvoraussetzung in $L_G$ liegt.

          \item $v_1 = v_{n+2}$ und damit wurde in der Ableitung von $v$ eine
            der folgenden Regeln benutzt:
            \begin{align*}
              (S, aSa), (S, bSb), (S, cSc)
            \end{align*}
        \end{itemize}
        Damit gilt $v \in L_G$.

      \item Induktionsanfang: Sei $w \in L_G$ und $\ell(w) = 0$, also $w =
        \varepsilon$, dann ist $w$ nach Definition von Palindromen
        (Induktionsanfang) ein Palindrom.

        Sei $v \in L_G$ und $\ell(v) = 1$, dann ist auch $v$ nach Definition von
        Palindromen (Induktionsanfang) ein Palindrom.

        Induktionsvoraussetzung: sei $w \in L_G$ und $\ell(w) = n$, dann ist $w$
        ein Palindrom.

        Induktionsschritt: sei $v = v_1 v_2 \ldots v_{n+2} \in L_G$ mit $\ell(v)
        = n+2$. Da $v \in L_G$ und $\ell(v) \geq 2$ ist, muss bei der Ableitung
        von $v$ eine der folgenden Ableitungsregeln angewandt worden sein:
        \begin{align*}
          (S, aSa), (S, bSb), (S, cSc)
        \end{align*}
        Daraus folgt $v_1 = v_{n+2}$.

        Da $v \in L_G$, ist das Teilwort $w = v_2 \ldots v_{n+1}$ auch in $L_G$.
        Weiterhin gilt $\ell(w) = n$, daher ist $w$ nach Induktionsvoraussetzung
        ein Palindrom.

        Insgesamt erhalten wir: $v$ ist ein Palindrom.
    \end{enumerate}

  \item
    \begin{enumerate}[(a)]
      \item Sei $G = (X, H, E, S)$ mit
        \begin{align*}
          X & = \{a, b\}\\
          H & = \{S, B\}\\
          E & = \{(S, \varepsilon), (S, abB), (S, B), (B, \varepsilon), (B, bbB)\}
        \end{align*}

      \item $S \rightarrow abB \rightarrow abbbB \rightarrow abbbbbB \rightarrow
        abbbbb$ und $S \rightarrow B \rightarrow bbB \rightarrow bbbbB
        \rightarrow bbbbbbB \rightarrow bbbbbb$.

      \item Sei $A = (X, Q, q_0, \delta, F)$ der DEA, der $L$ akzeptiert. Es gilt:
        \begin{align*}
          X & = \{a, b\}\\
          Q & = \{q_0, q_1, q_2, q_3, q_4, q_5\}\\
          F & = \{q_0, q_2\}
        \end{align*}
        Weiterhin ist $\delta: Q \times X \rightarrow Q$ wie folgt definiert:
        \begin{center}
          \begin{tabular}{lcc}
            \toprule
                  & $a$ & $b$\\
            \midrule
            $q_0$ & $q_1$ & $q_3$\\
            $q_1$ & $q_4$ & $q_2$\\
            $q_2$ & $q_4$ & $q_3$\\
            $q_3$ & $q_4$ & $q_2$\\
            $q_4$ & $q_4$ & $q_4$\\
            \bottomrule
          \end{tabular}
        \end{center}
  \end{enumerate}
\end{enumerate}
