\begin{enumerate}
  \item \begin{enumerate}[1.]
      \item Gegeben: sei $S$ eine Menge und $G_1, \ldots, G_n$ Teilmengen von
        $S$. Sei $\ell \in \N$.

        Frage: Gibt es eine Auswahl von Elementen $X \subseteq S$, sodass gilt
        $|X| \leq \ell$ und
        $G_i \cap X \neq \emptyset \ (\forall i \in \{1, \ldots, n\})$.

      \item \begin{enumerate}[(a)]
        \item Zunächst betrachten wir die Definition des Problems $\VC$:
          \begin{quote}
            Gegeben sei ein Graph $G = (V,E)$ und $k \in \N$.

            Das Entscheidungsproblem von $\VC$ lautet:
            gibt es eine Menge $U \subseteq V$ mit $|U| \leq k$ und
            $e \cap U \neq \emptyset$ für alle $e \in E$?
          \end{quote}
          $f$ tut folgendes:
          \begin{itemize}
            \item $V$ aus $\VC$ wird zu $S$ aus $\IV$
            \item eine Kante $e_i$ aus $\VC$ wird zu einer Menge $G_i$ aus $\IV$
              (für alle Kanten aus $E$)
            \item $k$ aus $\VC$ wird zu $\ell$ aus $\IV$
          \end{itemize}

          Oder anders gesagt:
          \begin{align*}
            f((G,k)) = f(((V, E), k)) = \begin{cases}
              S := V\\
              G_i := e_i\\
              k := \ell
            \end{cases}
          \end{align*}

        \item Wie aus den früheren Aufgabenserien bekannt, sei
          $\Sigma = \{1, 0, \#\}$.

          Der Graph
          $G = (V, E) = (\{v_1, \ldots, v_n\}, \{e_1, \ldots, e_m\})$
          und $k \in \N$
          werden auf dem Band einer Turingmaschine wie folgt kodiert:
          \begin{align*}
            \underbrace{\bin(1) \# \bin(2) \# \ldots \# \bin(n)}_{\text{Knoten}}
            \# \#
            \underbrace{%
              \underbrace{\bin(1) \# \bin(2)}_{\text{Kante\ } (v_1, v_2)}
              \#
              \underbrace{\bin(1) \# \bin(3)}_{\text{Kante\ } (v_1, v_3)}
              \#
              \ldots
            }_{\text{Kanten}}
            \# \#
            \bin(k)
          \end{align*}

        \item \begin{description}
          \item[„$\Rightarrow$“] Sei $U \subseteq V$ eine Lösung von $\VC$,
            d.\,h. $|U| \leq k$ und $\forall e \in E: e \cap U \neq \emptyset$.

            Wir transformieren das Problem wie in (a) beschrieben und setzen
            $X := U$.

            Dann gilt $|U| = |X| \leq \ell = k$ und
            $\forall i \in \{1, \ldots, n\}: G_i \cap X \neq \emptyset$,
            da $\forall e \in E: e \cap U \neq \emptyset$ gilt.

            Damit ist eine Lösung aus $\VC$ nach der Transformation durch $f$
            auch eine Lösung aus\\ $\IV$.

          \item[„$\Leftarrow$“] Sei $X \subseteq S$ eine Lösung von
            $f((G,k)) \in \IV$, d.\,h. $|X| \leq \ell$ und
            $\forall i \in \{1, \ldots, n\}: G_i \cap X \neq \emptyset$.

            Dann ist die Lösung von $\VC$: $U := X$.

            Es gilt $|X| = |U| \leq k = \ell$ und
            $\forall e \in E: e \cap U \neq \emptyset$,
            da $\forall i \in \{1, \ldots, n\}: G_i \cap X \neq \emptyset$ gilt.

            Damit ist eine durch $f$ transformierte Lösung aus $\IV$ auch eine
            Lösung aus $\VC$.
        \end{description}

        \item Da bei der Transformation durch $f$ keinerlei Umformungen, sondern
          einfach nur Ersetzungen stattfinden, liegt der Aufwand in $O(n)$ und
          ist damit polynomiell.
      \end{enumerate}
    \end{enumerate}

    \item \begin{enumerate}
      \item Im Namen.\footnote{höhö, kleiner Scherz}

        Während beim Problem $\CLIQUE$ die größte Clique gesucht wird (und damit
        das $k$ zur Problembeschreibung gehört), ist das $k$ bei $k-\CLIQUE$
        fest gewählt.

      \item Ein Algorithmus, der einfach alle möglichen Untergraphen
        durchprobiert, hat die Komplexität $O(n^k \cdot k^2)$.

        Dies ergibt sich wie folgt: sei $n = |V|$. Von den $n$ Knoten müssen
        alle Untergraphen der Größe $k$ durchprobiert werden. Dafür gilt
        \begin{align*}
          \binom{n}{k} & = \frac{n!}{k! \cdot (n-k)!}
          = \frac{%
            \overbrace{n \cdot (n-1) \cdots (n-k+1)}^{k \text{\ Faktoren}}
            }{k!}
          \in O(n^k)
        \end{align*}
        Jeder dieser Untergraphen hat maximal $\binom{k}{2} \in O(k^2)$ Kanten.

        Insgesamt landen wir bei einer Komplexität von $O(n^k \cdot k^2)$.
      \end{enumerate}

    \item
      \begin{enumerate}
        \item Greedy-Algorithmus für $\IS$:
          \begin{algorithmic}[1]
            \REQUIRE{Graph $G = (V,E)$}
            \STATE{$U \leftarrow \emptyset$}
            \STATE{$k \leftarrow 0$}
            \WHILE{$V \neq \emptyset$}
              \STATE{$k \leftarrow$ Maximum der Anzahl ausgehender Kanten der Knoten aus $V$}
              \STATE{$v \leftarrow$ ein zufälliger Knoten aus $V$, der $k$ ausgehende Kanten hat}
              \STATE{$U \leftarrow U \cup \{ v \}$}
              \STATE{entferne $v$ und alle benachbarten Knoten von $v$ aus $V$}
            \ENDWHILE%
            \RETURN{eine unabhängige Menge $U$ von $G$}
          \end{algorithmic}
          Die Menge $U$ ist unabhängig, weil in Zeile 7 für jeden hinzugefügten
          Knoten $v$ alle benachbarten Knoten von $v$ aus $V$ entfernt werden.
          Damit können keine zwei Knoten aus $U$ benachbart sein, sodass $U$
          eine unabhängige Menge sein muss.

        \item Seien $n = |V|$ und $k = |E|$.

          Im schlimmsten Fall (nämlich dann, wenn $G$ schon eine unabhängige
          Menge ist) wird in der while-Schleife immer nur ein einziger Knoten
          aus $V$ entfernt. Die Schleife würde dann $n$-mal laufen.

          Um das Maximum der Anzahl ausgehender Kanten aller Knoten aus $V$ zu
          bestimmen, muss diese Anzahl für jeden Knoten bestimmt werden (liegt in
          $O(n \cdot k)$) und dabei das Maximum gespeichert werden. Letzteres
          können wir schon bei der Bestimmung erledigen, sodass die Komplexität
          nicht größer wird.

          Die Bestimmung von $v$ in Zeile 5 liegt in $O(n)$: wir wählen einfach
          zufällig einen Knoten aus $V$ aus und überprüfen, ob dieser $k$
          ausgehende Kanten hat. Wenn nicht, versuchen wir es nochmal.

          $U \leftarrow U \cup \{ v \}$ liegt in $O(1)$.

          Das Entfernen von $v$ und dessen Nachbarschaft aus $V$ liegt im
          schlimmsten Fall (nämlich wenn $v$ mit allen anderen Knoten verbunden
          ist) in $O(n)$.

          Insgesamt liegt die Komplexität des Algorithmus’ in
          $O(n \cdot n \cdot k) = O(n^2 \cdot k)$.

        \item\ \\
          \begin{center}
            \begin{tikzpicture}[scale=1, auto]
              \node[vertex] (d) at (0,0) {$d$};
              \node[vertex] (c) at (6,0) {$c$};
              \node[vertex] (b) at (6,6) {$b$};
              \node[vertex] (a) at (0,6) {$a$};

              \node[vertex] (e) at (2,4) {$e$};
              \node[vertex] (f) at (4,4) {$f$};
              \node[vertex] (g) at (4,2) {$g$};
              \node[vertex] (h) at (2,2) {$h$};

              \path[edge] (a) -- (b) -- (c) -- (d) -- (a);
              \path[edge] (a) -- (e) -- (h) -- (g) -- (c);
              \path[edge] (e) -- (f) -- (b);
              \path[edge] (f) -- (g);
            \end{tikzpicture}
          \end{center}

        \pagebreak
        \item 1. Lauf:
          \begin{description}
            \item[Anfang:] $k \leftarrow 0$, $U \leftarrow \emptyset$
              \begin{center}
                \begin{tikzpicture}[scale=1, auto]
                  \node[vertex] (d) at (0,0) {$d$};
                  \node[vertex] (c) at (6,0) {$c$};
                  \node[vertex] (b) at (6,6) {$b$};
                  \node[vertex] (a) at (0,6) {$a$};

                  \node[vertex] (e) at (2,4) {$e$};
                  \node[vertex] (f) at (4,4) {$f$};
                  \node[vertex] (g) at (4,2) {$g$};
                  \node[vertex] (h) at (2,2) {$h$};

                  \path[edge] (a) -- (b) -- (c) -- (d) -- (a);
                  \path[edge] (a) -- (e) -- (h) -- (g) -- (c);
                  \path[edge] (e) -- (f) -- (b);
                  \path[edge] (f) -- (g);
                \end{tikzpicture}
              \end{center}
            \item[Bestimme $k$:] $k \leftarrow 3$
            \item[Wahl von $v$:] $v \leftarrow a$,
              damit ist $U = \{ a \}$
            \item[Entferne $v$ und Nachbarn von $v$:]\ \\
              \begin{center}
                \begin{tikzpicture}[scale=1, auto]
                  \node[vertex] (c) at (6,0) {$c$};

                  \node[vertex] (f) at (4,4) {$f$};
                  \node[vertex] (g) at (4,2) {$g$};
                  \node[vertex] (h) at (2,2) {$h$};

                  \path[edge] (h) -- (g) -- (c);
                  \path[edge] (f) -- (g);
                \end{tikzpicture}
              \end{center}
            \item[Bestimme $k$:] $k \leftarrow 3$
            \item[Wahl von $v$:] $v \leftarrow g$,
              damit ist $U = \{ a, g \}$
            \item[Entferne $v$ und Nachbarn von $v$:] wir erhalten
              $V = \emptyset$

            \item[Ergebnis: ] Es wird $U = \{ a, g \}$ zurückgegeben, was eine
              optimale Lösung darstellt.
          \end{description}

          \pagebreak
          2. Lauf:
          \begin{description}
            \item[Anfang:] $k \leftarrow 0$, $U \leftarrow \emptyset$
              \begin{center}
                \begin{tikzpicture}[scale=1, auto]
                  \node[vertex] (d) at (0,0) {$d$};
                  \node[vertex] (c) at (6,0) {$c$};
                  \node[vertex] (b) at (6,6) {$b$};
                  \node[vertex] (a) at (0,6) {$a$};

                  \node[vertex] (e) at (2,4) {$e$};
                  \node[vertex] (f) at (4,4) {$f$};
                  \node[vertex] (g) at (4,2) {$g$};
                  \node[vertex] (h) at (2,2) {$h$};

                  \path[edge] (a) -- (b) -- (c) -- (d) -- (a);
                  \path[edge] (a) -- (e) -- (h) -- (g) -- (c);
                  \path[edge] (e) -- (f) -- (b);
                  \path[edge] (f) -- (g);
                \end{tikzpicture}
              \end{center}
            \item[Bestimme $k$:] $k \leftarrow 3$
            \item[Wahl von $v$:] $v \leftarrow b$,
              damit ist $U = \{ b \}$
            \item[Entferne $v$ und Nachbarn von $v$:]\ \\
              \begin{center}
                \begin{tikzpicture}[scale=1, auto]
                  \node[vertex] (d) at (0,0) {$d$};

                  \node[vertex] (e) at (2,4) {$e$};
                  \node[vertex] (g) at (4,2) {$g$};
                  \node[vertex] (h) at (2,2) {$h$};

                  \path[edge] (e) -- (h) -- (g);
                \end{tikzpicture}
              \end{center}
            \item[Bestimme $k$:] $k \leftarrow 2$
            \item[Wahl von $v$:] $v \leftarrow h$,
              damit ist $U = \{ b, h \}$
            \item[Entferne $v$ und Nachbarn von $v$:] wir erhalten
              \begin{center}
                \begin{tikzpicture}[scale=1, auto]
                  \node[vertex] (d) at (0,0) {$d$};
                \end{tikzpicture}
              \end{center}
            \item[Bestimme $k$:] $k \leftarrow 0$
            \item[Wahl von $v$:] $v \leftarrow d$,
              damit ist $U = \{ b, h, d \}$
            \item[Entferne $v$ und Nachbarn von $v$:] wir erhalten
              $V = \emptyset$

            \item[Ergebnis: ] Es wird $U = \{ b, h, d \}$ zurückgegeben, was
              keine optimale Lösung darstellt.
          \end{description}

          Bei diesen zwei Läufen sehen wir, dass eine optimale Lösung gefunden
          werden kann, dies aber nicht immer passiert.
      \end{enumerate}
\end{enumerate}
