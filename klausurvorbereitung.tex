\documentclass[
  a4paper,
  11pt,
]{scrartcl}

\usepackage[utf8]{inputenc}
\usepackage[
  cm,
  headings
]{fullpage}
\usepackage[ngerman]{babel}
\usepackage{amsmath}
\usepackage{amssymb}
% Für die Klammern der Interpretations-abbildung
\usepackage{stmaryrd}

\usepackage{color}
\definecolor{mygray}{rgb}{0.5,0.5,0.5}
\definecolor{myblue}{rgb}{0,0,0.8}

\usepackage[
  colorlinks=true,
  linkcolor=myblue,
]{hyperref}

% Coole Zeichnungen:
\usepackage{tikz}
\tikzstyle{vertex}=[draw, circle, minimum size=20pt]
\tikzstyle{edge}=[draw, -]
\usetikzlibrary{%
  %backgrounds,
  %mindmap,
  %shapes.geometric,
  %shapes.symbols,
  %shapes.misc,
  %shapes.multipart,
  positioning,
  %fit,
  %calc,
  %arrows,
  automata,
  %trees,
  %decorations.pathreplacing,
}

\usepackage{pgfplots}

% Für Zeilenumbrüche ohne Indentations
\setlength{\parindent}{0pt}

\usepackage{listings}
\lstset{%
  basicstyle=\ttfamily,
  keywordstyle=\color{blue},
  commentstyle=\color{mygray},
  language=C,
  showstringspaces=false,
}

\usepackage{enumerate}
% Schönere Tabellen
% dazu gibt's neue Kommandos:
% - \toprule[(Dicke)], \midrule[(Dicke)], \bottomrule[(Dicke)]
% - \addlinespace: Extrahöhe zwischen Zeilen
\usepackage{booktabs}
% um in Tabellen eine Zelle über mehrere Zeilen laufen zu lassen:
\usepackage{multirow}

\usepackage{algorithmic}
\renewcommand{\algorithmicrequire}{\textbf{Eingabe:}}

% coole Kopf- und Fußzeilen:
\usepackage{fancyhdr}
% Seitenstil ist natürlich fancy:
\pagestyle{fancy}
% alle Felder löschen:
\fancyhf{}

%\fancyhead[L]{
%}
%\fancyhead[R]
%\fancyfoot[L]{}

\fancyfoot[C]{\thepage}

\title{Klausurvorbereitung: Theoretische Informatik}

\subtitle{Universität Rostock}

%\author

\date{}

\newcommand{\p}{\leq_{\textsf{p}}}
\newcommand{\N}{\mathbb{N}}
\renewcommand{\O}{\mathcal{O}}
\renewcommand{\ae}{\leq_{\textsf{ae}}}

\newcommand{\COLOR}{\textsf{COLOR}}
\newcommand{\COLORDrei}{\textsf{COLOR-3}}
\newcommand{\DreiSAT}{\textsf{3SAT}}
\newcommand{\bin}{\text{bin}}
\newcommand{\TSP}{\textsf{TSP}}
\newcommand{\HC}{\textsf{HAMILTONIAN CIRCUIT}}
\newcommand{\TSPOPTI}{\textsf{TSP\_OPT\_1}}
\newcommand{\NELP}{\textsf{0--1 LINEAR PROGRAMMING}}
\newcommand{\CLIQUE}{\textsf{CLIQUE}}
\newcommand{\NP}{\textsf{NP}}
\newcommand{\id}{\text{id}}
\newcommand{\IS}{\textsf{INDEPENDENT SET}}
\newcommand{\VC}{\textsf{VERTEX COVER}}
\newcommand{\IV}{\textsf{INTERESSENVERTRETER}}

\begin{document}

\maketitle

\tableofcontents

\section{Themen \& Inhalte}
\label{sec:themen}

\begin{itemize}
  \item \NP-Vollständigkeit durch Polynomialreduktion zeigen
    \begin{itemize}
      \item Bsp: $\DreiSAT \p \IS$
    \end{itemize}
  \item Zeigen, dass eine Sprache nicht regulär ist (Pumping-Lemma für reguläre
    Sprachen)
    \begin{itemize}
      \item Bsp: $L = \{ 0^n 1^n \mid n \in \N \}$ ist nicht regulär
    \end{itemize}
  \item Turing-Maschine bauen (deterministisch und nichtdeterministisch)
    \begin{itemize}
      \item Bsp: eine NTM mit $\Sigma = \{\Box, 0, 1\}$, die zwei Einsen auf dem
        Band erkennt (Startposition ist beliebig)
    \end{itemize}
  \item Mengenschreibweise einer Sprache aus einer gegebenen Grammatik angeben
    \begin{itemize}
      \item siehe Serie 8, Aufgabe 3 und Aufgabe 4
    \end{itemize}
  \item nicht-/deterministischen endlichen Automaten aus einer gegebenen
    Grammatik konstruieren
    \begin{itemize}
      \item siehe Serie 8, Aufgabe 3 und Aufgabe 4
      \item dazu gehört auch: eine rechtslineare Grammatik normalisieren
    \end{itemize}
  \item deterministischen endlichen Automaten minimieren
    \begin{itemize}
      \item siehe Serie 9, Aufgabe 3
      \item siehe \hyperref[sub:serie_10]{Serie 10},
        \hyperref[kfs_serie_10_aufgabe_1]{Aufgabe 1}
    \end{itemize}
  \item Pumpingzahl einer regulären Sprache angeben
    \begin{itemize}
      \item siehe \hyperref[sub:serie_10]{Serie 10},
        \hyperref[kfs_serie_10_aufgabe_2_c]{Aufgabe 2 (c)}
    \end{itemize}
  \item NDEA $\rightarrow$ DEA
    \begin{itemize}
      \item siehe Serie 9, Aufgabe 1
    \end{itemize}
\end{itemize}

\section{Übungsserien}


\subsection{Serie 1}
\label{sub:serie_1}

\begin{enumerate}
  \item
    \begin{enumerate}
      \item Die TM überprüft die \textsc{Palindrom}-Eigenschaft rekursiv, indem
        es das erste und das letzte Zeichen auf Gleichheit überprüft. Bei
        Gleichheit werden das erste und letzte Zeichen bei diesem Vorgang
        entfernt und der Vorgang beginnt erneut.

        Wörter der Länge $< 2$ sind dabei immer Palindrome.

      \item Wir betrachten das Wort $abaaba$ der Länge $n = 6$.

        Der Algorithmus der oben definierten TM arbeiten in vier „Phasen“, die
        jeweils unterschiedlich viele Schritte (Kopfbewegungen) benötigen.
        Wir nehmen an, dass der Kopf der TM zu Beginn am Anfang des Wortes
        steht.\footnote{Wir können die TM auch so verändern, dass der Kopf zu
        Beginn am Ende des Wortes oder sogar dahinter steht. Dazu müssen wir
        einfach eine zusätzliche Phase einbauen, die den Kopf zum Anfang des
        Wortes laufen lässt. Dies würde linearen Aufwand hinzufügen, was am Ende
        keinen Unterschied macht.}

        \begin{tabular}{cll}
          Phase & Beschreibung & benötigte Schritte\\
          \midrule
          I & erstes Zeichen lesen & $1$\\
          II & zum letzten Zeichen gehen & $6 = n$\\
          III & letztes Zeichen lesen & $1$\\
          IV & zum ersten Zeichen gehen & $5 = n-1$\\
          \midrule
          & & $13 = 2n+1$\\
          \bottomrule
        \end{tabular}

        Es bleibt das Wort $baab$ der Länge $4$ übrig. Der Algorithmus beginnt
        erneut und benötigt $2(n-2) + 1 = 9$ Schritte. Für das Wort $aa$ der
        Länge $2$ werden $2(n-4) + 1 = 5$ Schritte und für das leere Wort wird
        $1$ Schritt benötigt („springe in den akzeptierenden Zustand“).

        Um die Gleichheit des ersten und letzten Buchstabens eines Wortes der
        Länge $n$ zu überprüfen, werden $2n+1$ Schritte benötigt. Dies
        funktioniert auch für ein Wort der Länge $1$:

        \begin{tabular}{cll}
          Phase & Beschreibung & benötigte Schritte\\
          \midrule
          I & erstes Zeichen lesen & $1$\\
          II & zum letzten Zeichen gehen & $1$\\
          III' & in den akzeptierenden Zustand springen & $1$\\
          \midrule
          & & $3$\\
          \bottomrule
        \end{tabular}

        Die Anzahl der insgesamt benötigten Schritte berechnet sich durch
        folgende Summenformel:
        \begin{align*}
          \sum\limits_{k=0}^{\lfloor n/2 \rfloor} 2(n - 2k) + 1
        \end{align*}

        Für ein Wort der Länge $6$ werden insgesamt $13 + 9 + 5 + 1 = 28$
        Schritte benötigt.

        Die obere Grenze für den Index wird abgerundet, damit die Summation auch
        für Wörter ungerader Länge funktioniert.

        Für die genaue Abschätzung betrachten wir zunächst die Anzahl der
        Einsen. Diese beträgt $\lfloor n/2 \rfloor + 1$, da der Index bei $0$
        startet. Somit erhalten wir:
        \begin{align*}
          \sum\limits_{k=0}^{\lfloor n/2 \rfloor} 2(n - 2k) + 1
          = \lfloor n/2 \rfloor + 1 + \sum\limits_{k=0}^{\lfloor n/2 \rfloor} 2(n - 2k)
        \end{align*}

        Den konstanten Faktor $2$ können wir ohne Probleme herausziehen:
        \begin{align*}
          \lfloor n/2 \rfloor + 1 + \sum\limits_{k=0}^{\lfloor n/2 \rfloor} 2(n - 2k)
          = \lfloor n/2 \rfloor + 1 + 2 \sum\limits_{k=0}^{\lfloor n/2 \rfloor} n - 2k
        \end{align*}

        Den Term $\sum\limits_{k=0}^{\lfloor n/2 \rfloor} n - 2k$ können wir in
        zwei Summen aufspalten:
        \begin{align*}
          \lfloor n/2 \rfloor + 1 + 2 \sum\limits_{k=0}^{\lfloor n/2 \rfloor} n - 2k
          = \lfloor n/2 \rfloor + 1 + 2
            \left(
              \sum\limits_{k=0}^{\lfloor n/2 \rfloor} n
              - 2 \sum\limits_{k=0}^{\lfloor n/2 \rfloor} k
            \right)
        \end{align*}

        Der Term
        $\sum\limits_{k=0}^{\lfloor n/2 \rfloor} n$
        vereinfacht sich zu dem Produkt
        $\left( \lfloor n/2 \rfloor + 1 \right) n$
        und der Term
        $\sum\limits_{k=0}^{\lfloor n/2 \rfloor} k$
        ist die Gaußsche Summenformel, also
        $\frac{\lfloor n/2 \rfloor \left(\lfloor n/2 \rfloor + 1\right)}{2}$.

        Damit landen wir bei
        \begin{align*}
          \lfloor n/2 \rfloor + 1 + 2
            \left(
              \sum\limits_{k=0}^{\lfloor n/2 \rfloor} n
              - 2 \sum\limits_{k=0}^{\lfloor n/2 \rfloor} k
            \right)
          & = \lfloor n/2 \rfloor + 1 + 2
            \left(
              \left( \lfloor n/2 \rfloor + 1 \right) n
              - 2
              \frac{\lfloor n/2 \rfloor \left(\lfloor n/2 \rfloor + 1\right)}{2}
            \right)\\
          & = \lfloor n/2 \rfloor + 1 + 2
            \left( \lfloor n/2 \rfloor + 1 \right)
            \left( n - \lfloor n/2 \rfloor \right)
        \end{align*}

        Wir wählen für die weitere Analyse die Abschätzung
        $\lfloor n/2 \rfloor \leq \frac{n}{2}$
        und erhalten:
        \begin{align*}
          \lfloor n/2 \rfloor + 1 + 2
            \left( \lfloor n/2 \rfloor + 1 \right)
            \left( n - \lfloor n/2 \rfloor \right)
          & \leq \frac{n}{2} + 1 + 2
            \left( \frac{n}{2} + 1 \right)
            \left( n - \frac{n}{2} \right)\\
          & = \frac{1}{2} n^2 + \frac{3}{2} n + 1
            = \frac{1}{2} (n+2) (n+1)
            = \binom{n+2}{2}
        \end{align*}
        Eine kombinatorische Begründung fällt mir dafür aber nicht ein.

      \item --- % Es gilt $\frac{1}{2} n^2 + \frac{3}{2} n + 1 \in O(n^2)$.
        % Ist wohl falsch: „Laut VL: worst case des besten Programmes (??) O(n)“
        % Dabei kann ich (??) nicht entziffern.

      \item ---
    \end{enumerate}

    \item \begin{enumerate}
      \item Ist korrekt. Es ist zu zeigen, dass für fast alle $n$ und ein
        beliebiges $k \in \N \setminus \{ 0 \}$ gilt:
        \begin{align*}
          3n + 4 + \log n \leq k \cdot n
        \end{align*}
        Wir wählen $k := 5$ und erhalten:
        \begin{align*}
          3n + 4 + \log n & \leq 5n\\
          4 + \log n & \leq 2n = n + n
        \end{align*}
        Es gilt $4 \leq n$ für fast alle $n$ (nämlich ab $n > 4$).

        Bleibt zu zeigen, dass $\log n \leq n$.
        Wir wenden die $e$-Funktion auf beiden Seiten an und erhalten:
        \begin{align*}
          n \leq e^n = \sum\limits_{i=0}^{\infty} \frac{n^i}{i!} = 1 + n + \dots
        \end{align*}
        Daraus folgt $n \leq 1 + n$ für alle $n \in \N$. \hfill $\blacksquare$

      \item Ist korrekt. Es ist zu zeigen, dass für fast alle $n$ und ein
        beliebiges $k \in \N \setminus \{ 0 \}$ gilt:
        \begin{align*}
          \log n \leq k \cdot \sqrt{n}
        \end{align*}
        Wir wählen $k := 2$ und erhalten:
        \begin{align*}
          \log n & \leq 2 \sqrt{n}\\
          n & \leq e^{2 \sqrt{n}} \geq 1 + 2 \sqrt{n} + \frac{2 n}{2}\\
          n & \leq 1 + 2 \sqrt{n} + n\\
          0 & \leq 1 + 2 \sqrt{n}
        \end{align*}
        Die letzte Aussage ist für alle $n \in \N$ wahr. \hfill $\blacksquare$

      \item Ist nicht korrekt. Sei $k \in \N \setminus \{ 0 \}$ beliebig, aber
        fest gewählt.

        Es gilt
        \begin{align*}
          \sum\limits_{i = 1}^{n} i = \frac{n(n+1)}{2}
        \end{align*}
        Es wäre zu zeigen, dass für fast alle $n \in \N$ gilt:
        \begin{align*}
          \frac{n(n+1)}{2} & \leq k \cdot n\\
          n(n+1) & \leq 2k \cdot n
        \end{align*}
        Für $n > 0$ müsste gelten:
        \begin{align*}
          n+1 & \leq 2k
        \end{align*}
        Ab $n \geq 2k$ ist die Ungleichung aber nicht mehr erfüllt. Also gilt
        \begin{align*}
          \sum\limits_{i = 1}^{n} i \notin O(n)
        \end{align*}
        \ \hfill $\blacksquare$
    \end{enumerate}

    \item ---

    \item Die Komplexität des Suchalgorithmus „binäre Suche“ liegt in
      $O(\log n)$ bzw.\ genauer gesagt in $O(\log_2 n)$, da das Intervall, in
      welchem das Schlüsselement gesucht wird, bei jedem rekursiven Aufruf
      mindestens um die Hälfte verringert wird.

      Man könnte den Verlauf der binären Suche als Binärzahl interpretieren.
      Jede Stelle gibt entweder „suche links weiter“ ($0$) oder „suche rechts
      weiter“ ($1$) an. Im schlimmsten Fall wird immer nur links oder rechts
      gesucht. Falls immer nur rechts weitergesucht wird, erhalten wir die
      Binärdarstellung von $N-1$ und diese enthält $\log_2 (N-1)$ Einsen. Falls
      immer nur links weitergesucht wird, erhalten wir $\log_2 (N-1)$ Nullen.
\end{enumerate}



\subsection{Serie 2}
\label{sub:serie_2}

\begin{enumerate}
  \item
    \begin{enumerate}[(a)]
      \item Sei $\Sigma$ ein beliebiges nichtleeres Alphabet. Es ist $L \in P$,
        genau dann wenn $L \subseteq \Sigma^*$ und eine Turingmaschine $M$
        existiert, die die char. Funktion
        $\chi_L: \Sigma^* \rightarrow \{0, 1\}$
        in polynomieller Zeit berechnet. Im Endzustand von $M$ steht auf dem
        Band demnach:
        \begin{align*}
          M(\omega) = \begin{cases}
            1 & \omega \in L\\
            0 & \omega \notin L
          \end{cases}
        \end{align*}
        Dazu betrachten wir die Turingmaschine $M'$, die im Endzustand auf dem
        Band folgendes Ergebnis liefert:
        \begin{align*}
          M'(\omega) = \begin{cases}
            1 & M(\omega) = 0\\
            0 & M(\omega) = 1
          \end{cases}
        \end{align*}
        Die Turingmaschine $M'$ berechnet also in polynomieller Zeit, ob $\omega
        \in \Sigma^* \setminus L$ ist. Daraus folgt $P \subseteq coP$.

        Analog dazu betrachten wir $L \in coP$: es ist $L \in coP$ genau dann,
        wenn $L \subseteq \Sigma^*$ und eine Turingmaschine $M$ existiert, die die
        char. Funktion
        $\chi_L: \Sigma^* \setminus L \rightarrow \{0, 1\}$
        in polynomieller Zeit berechnet. Im Endzustand von $M$ steht auf dem
        Band demnach:
        \begin{align*}
          M(\omega) = \begin{cases}
            1 & \omega \in \Sigma^* \setminus L\\
            0 & \omega \notin \Sigma^* \setminus L
          \end{cases}
        \end{align*}
        Wir definieren $M'$ wie oben und erhalten das Ergebnis, dass $M'$ in
        polynomieller Zeit berechnet, ob $\omega \in L$ ist. Daraus folgt $coP
        \subseteq P$ und insgesamt $P = coP$.

      \item Sei $L_1 \in P$ und $L_2 \in P$. Das heißt, dass die Turingmaschinen
        $M_1$ und $M_2$ existieren, die $L_1$ bzw. $L_2$ in polynomieller Zeit
        entscheiden. Der Aufwand von $M_1$ für $L_1$ sei in $O(n^{k_1})$ und der
        Aufwand von $M_2$ für $L_2$ in $O(n^{k_2})$ (für konstante Werte $k_1$
        und $k_2$).

        Die Konkatenation $L_1 L_2$ ist wie folgt definiert:
        \begin{align*}
          L_1 L_2 = \{ \omega_1 \omega_2 \mid \omega_1 \in L_1, \omega_2 \in L_2
          \}
        \end{align*}
        Wir definieren eine Turingmaschine $M$, die $L_1 L_2$ in polynomieller
        Zeit entscheidet. Diese muss ein Eingabewort $\omega$ jedoch zunächst
        so in $\omega_1$ und $\omega_2$ aufspalten, dass $\omega_1 \in L_1$ und
        $\omega_2 \in L_2$ gilt.

        Dies erfolgt nach folgendem Pseudocode:
        \begin{algorithmic}[1]
          \REQUIRE{$\omega = a_1 a_2 \ldots a_n$}
          \FOR{$i = 0$ \textbf{to} $n$}\label{alg:schleife}
          \STATE{teile $\omega$ in $\omega_1 = a_1 \ldots a_i$ und $\omega_2 = a_{i+1} \ldots a_n$}\label{alg:aufteilung}
          \STATE{berechne Ergebnis von $M_1$ mit der Eingabe $\omega_1$}
          \STATE{berechne Ergebnis von $M_2$ mit der Eingabe $\omega_2$}
          \IF{$M_1$ und $M_2$ akzeptieren ihre Eingaben}
            \STATE{akzeptiere $\omega$}
          \ENDIF{}
          \ENDFOR{}
          \IF{keine Aufteilung von $\omega$ in $\omega_1$ und $\omega_2$ führte
          in einen akzeptierenden Zustand von $M_1$ und $M_2$}
            \STATE{lehne $\omega$ ab}
          \ENDIF{}
        \end{algorithmic}
        Bei der Aufteilung von $\omega$ in $\omega_1$ und $\omega_2$ in
        Zeile~\ref{alg:aufteilung} gilt
        $a_1 \ldots a_0 = \epsilon = a_{n+1} \ldots a_n$.

        Es bleibt zu zeigen, dass $M$ die Eingabe $\omega$ in polynomieller Zeit
        akzeptiert. Wie schon erwähnt liegt der Aufwand von $M_1$ für $L_1$ in
        $O(n^{k_1})$ und der Aufwand von $M_2$ für $L_2$ in $O(n^{k_2})$.
        Insgesamt liegt der Aufwand für einen Schleifendurchlauf in $O(n^{k_1})
        + O(n^{k_2}) = O(n^k)$ mit $k = \max(k_1, k_2)$. Dazu kommt noch der
        Aufwand für die Aufteilung von $\omega$ in $\omega_1$ und $\omega_2$.
        Die Schleife in Zeile~\ref{alg:schleife} wird dabei maximal $n+1$ mal
        durchlaufen. Das heißt, dass der Aufwand für die Entscheidung von $L_1
        L_2$ gleich $(n+1) \cdot O(n^k)$ ist und damit in $O(n^{k+1})$ liegt,
        was polynomiellem Aufwand entspricht. Es gilt also $L_1 L_2 \in P$.

      \item —
    \end{enumerate}

  \item
    \begin{enumerate}[(a)]
    \item Gegeben sei ein ungerichteter Graph $(V,E)$.

      Finde das minimale $k$, sodass man die Knoten mit $k$ Farben so färben
        kann, dass benachbarte Knoten immer verschiedene Farben haben?

    \item Gegeben sei ein ungerichteter Graph $(V,E)$ mit $V = \{v_1, \ldots,
        v_n \}$. Wir können den Graphen mit maximal $n$ Farben färben und falls
        der Graph nur einen Knoten besitzt, können wir ihn mit genau einer Farbe
        färben. Der minimale Wert für $k$ befindet sich demnach im Intervall
        $[1, n]$.

        Wir nehmen an, dass der Algorithmus $\COLOR(V, E, k)$ in
        deterministischer Polynomialzeit $p(n)$ entscheiden kann, ob der
        ungerichtete Graph $(V, E)$ mit $k$ Farben gefärbt werden kann.

        Folgender Algorithmus löst die Optimierungsvariante I von \COLOR{}:
        \begin{algorithmic}
          \REQUIRE{ein ungerichteter Graph $(V, E)$}
          \STATE{min $\leftarrow 1$}
          \STATE{max $\leftarrow |V|$}
          \WHILE{min $<$ max}
            \STATE{mid $\leftarrow \lfloor (\text{min} + \text{max})/2 \rfloor$}
            \IF{$\COLOR(V, E, \text{mid})$}
              \STATE{max $\leftarrow$ mid}
            \ELSE{}
              \STATE{min $\leftarrow$ mid}
            \ENDIF{}
          \ENDWHILE{}
          \RETURN{min}
        \end{algorithmic}

        Diese Binärschachtelung benötigt maximal
        $\lfloor \text{log}_2 \, n \rfloor + 1$
        Schritte.

        Der Graph wird auf dem Band einer Turingmaschine mit dem Bandalphabet
        $\{0, 1, \#\}$ wie folgt kodiert:
        \begin{align*}
          \underbrace{\bin(1) \# \bin(2) \# \ldots \# \bin(n)}_{\text{Knoten}}
          \# \#
          \underbrace{%
            \underbrace{\bin(1) \# \bin(2)}_{\text{Kante\ } (v_1, v_2)} \# \# \ldots
          }_{\text{Kanten}}
        \end{align*}
        Die Länge der binärkodierten Eingabe ergibt sich wie folgt:
        \begin{itemize}
          \item ein binärkodierter Knoten hat maximal die Länge
            $\bin(n) \leq \lfloor \text{log}_2\, n \rfloor + 1$
          \item davon haben wir $n$ Stück
          \item dazu kommen $n-1$ Trennzeichen zwischen den Knoten
          \item[$\Rightarrow$] maximal
            $n (\lfloor \text{log}_2\, n \rfloor + 1) + n-1
            \leq n (\lfloor \text{log}_2\, n \rfloor + 2)$
            Zeichen für die Knoten
          \item in einem Graphen existieren maximal
            $\binom{n}{2} = \frac{n (n-1)}{2}$
            Kanten
          \item eine binärkodierte Kante besteht aus $2$ Knoten, einem
            Trennzeichen zwischen den Knoten und zwei Trennzeichen vor dem
            ersten Knoten
          \item[$\Rightarrow$] maximal $\frac{n (n-1)}{2}
            \cdot \left(2
            \cdot (\lfloor \text{log}_2\, n \rfloor + 1) + 1 + 2\right)
            \leq n^2 (2 \lfloor \text{log}_2\, n \rfloor + 5)$
            Zeichen für die Kanten
        \end{itemize}
        Insgesamt liegt der Aufwand demnach in
        $O\Big(
          \big(
            n (\lfloor \text{log}_2\, n \rfloor + 2)
            + n^2 (2 \lfloor \text{log}_2\, n \rfloor + 5)
          \big) \cdot p(n)
        \Big)$, was die Polynomialzeiteigenschaft zeigt.
    \end{enumerate}

  \item
    \begin{enumerate}[(a)]
      \item Optimierungsvariante I für \TSP{}:

        Gegeben sei ein vollständiger Graph $(V, E = V \times V)$ mit
        einer Kostenfunktion $c: E \rightarrow \N$.

        Finde die minimalen Kosten einer Rundreise über alle Knoten.

      Optimierungsvariante II für \TSP{}:

        Gegeben sei ein vollständiger Graph $(V, E = V \times V)$ mit
        einer Kostenfunktion $c: E \rightarrow \N$.

        Finde die minimalen Kosten einer Rundreise über alle Knoten und einen
        Kreis über alle Knoten im Graphen, der diese Kosten erfüllt.

      \item
        \begin{enumerate}[1.]
          \item Wir nehmen an, dass der Algorithmus $\TSP(V, E, c, k)$ in
            deterministischer Polynomialzeit $q(n)$ entscheiden kann, ob in dem
            vollständigem Graphen $(V, E)$ mit der Kostenfunktion
            $c: E \rightarrow \N$ eine Rundreise über alle Knoten existiert,
            dessen Kosten $\leq k$ sind.

            Folgender Algorithmus löst die Optimierungsvariante I von \TSP{}:
            \begin{algorithmic}[1]
              \REQUIRE{ein vollständiger Graph $(V, E)$ und eine Kostenfunktion $c: E \rightarrow \N$}
              \STATE{lower $\leftarrow 0$}
              \STATE{$C \leftarrow \max\,\{c(e) \mid e \in E\}$}
              \STATE{upper $\leftarrow |V| \cdot C$}
              \WHILE{lower $<$ upper}
                \STATE{mid $\leftarrow \lfloor (\text{lower} + \text{upper})/2 \rfloor$}
                \IF{$\TSP(V, E, c, \text{mid})$}
                  \STATE{upper $\leftarrow$ mid}
                \ELSE{}
                  \STATE{lower $\leftarrow$ mid}
                \ENDIF{}
              \ENDWHILE{}
              \RETURN{lower}
            \end{algorithmic}

          \item Der Graph wird auf dem Band einer Turingmaschine mit dem
            Bandalphabet $\{0, 1, \#\}$ wie folgt kodiert:
            \begin{align*}
              \underbrace{\bin(1) \# \bin(2) \# \ldots \# \bin(n)}_{\text{Knoten}}
              \# \#
              \underbrace{%
                \underbrace{\bin(1) \# \bin(2)}_{\text{Kante\ } (v_1, v_2)}
                \#
                \underbrace{\bin(c((v_1, v_2)))}_{\text{Kosten von } (v_1, v_2)}
                \# \# \ldots
              }_{\text{Kanten}}
            \end{align*}
            Die Kodierung entspricht im Wesentlichen der von Aufgabe 2. (b), nur
            dass zu jeder Kante noch die Binärkodierung der Kostenfunktion
            hinzukommt.

            Sei $C$ der im Graphen größte angenommene Wert der Kostenfunktion.
            Ein Wert der Kostenfunktion nimmt dann maximal $\lfloor \text{log}_2
            C \rfloor + 1$ Stellen in Anspruch. Dazu kommt ein Trennzeichen pro
            Kante. Insgesamt kommen maximal $n^2 \cdot \left( \lfloor
            \text{log}_2 C \rfloor + 2 \right)$ Zeichen hinzu, sodass der
            Aufwand für die Optimierungsvariante I in
            \begin{align*}
              O\Big(
                \big(
                  n (\lfloor \text{log}_2\, n \rfloor + 2)
                  + n^2 (2 \lfloor \text{log}_2\, n \rfloor + 5)
                  + n^2 \left( \lfloor \text{log}_2 C \rfloor + 2 \right)
                \big) \cdot q(n)
              \Big)
            \end{align*}
            liegt, was die Polynomialzeiteigenschaft zeigt.
        \end{enumerate}

      \item
        \begin{enumerate}[1.]
          \item Wir nehmen an, dass der Algorithmus $\TSPOPTI(V, E, c)$ in
            deterministischer Polynomialzeit $r(n)$ in dem vollständigem Graphen
            $(V, E)$ mit der Kostenfunktion $c: E \rightarrow \N$ die minimalen
            Kosten einer Rundreise über alle Knoten berechnen kann.

            Folgender Algorithmus löst die Optimierungsvariante II von \TSP{}:
            \begin{algorithmic}[1]
              \REQUIRE{ein vollständiger Graph $(V, E)$ und eine Kostenfunktion $c: E \rightarrow \N$}
              \STATE{min $\leftarrow \TSPOPTI(V, E, c)$}
              \STATE{$U \leftarrow E$}
              \WHILE{$U \neq \emptyset$}
                \STATE{wähle $e$ aus $U$}
                \STATE{$U \leftarrow U \setminus \{ e \}$}
                \STATE{$k \leftarrow \TSPOPTI(V, E \setminus \{ e \}, c)$}
                \IF{$k =$ min}
                  \STATE{$E \leftarrow E \setminus \{ e \}$}
                \ENDIF{}
              \ENDWHILE{}
              \RETURN{$E$}
            \end{algorithmic}

          \item Dieser Algorithmus überprüft für jede Kante im Graphen, ob sie
            wichtig ist und zur Rundreise gehört oder ob die Rundreise ohne sie
            teurer würde. Am Ende bleiben nur die wichtigen Kanten übrig.

            Dabei wird jede Kante einmal überprüft (wovon maximal $n^2$
            existieren), sodass die Komplexität der Optimierungsvariante II in
            $O(n^2 \cdot r(n))$ liegt.
        \end{enumerate}
    \end{enumerate}
\end{enumerate}



\subsection{Serie 3}
\label{sub:serie_3}

\begin{enumerate}
  \item
    \begin{enumerate}[(a)]
      \item In der Übung wurde gezeigt, dass sich \DreiSAT{} auf das
        Halteproblem reduzieren lässt.

        Um zu zeigen, dass das Halteproblem wenigstens so schwer wie
        \COLORDrei{} ist, überführen wir \COLORDrei{} in \DreiSAT{}.

        Gegeben sei ein Graph $(V, E)$ mit $|V| = n$ und eine Menge mit drei
        Farben, die wir mit $\{1,2,3\}$ abstrahieren. Wir führen für jeden
        Knoten und jede Farbe ein Literal $v_{i, c}$ (mit $i \in \{1, \ldots,
        n\}$ und $c \in \{1,2,3\}$) ein, welches angibt, ob der Knoten mit
        dieser Farbe gefärbt wurde.

        Die aussagenlogische Formel für \DreiSAT{} ergibt sich aus zwei
        Teilformeln: $K \wedge N$

        Dabei gilt:
        \begin{itemize}
          \item $K$ sind die Bedingungen für die Färbungen der einzelnen Knoten,
          \item $N$ sind die Bedingungen für die unterschiedlichen Färbungen
            benachbarter Knoten
        \end{itemize}

        Ein einzelner Knoten darf nur mit einer Farbe gefärbt werden, also gilt:
        \begin{align*}
          K = \bigwedge\limits_{i \in \{1, \ldots, n\}}
            (v_{i, 1} \vee v_{i, 2} \vee v_{i, 3}) \wedge
            (\neg v_{i, 1} \vee \neg v_{i, 2}) \wedge
            (\neg v_{i, 1} \vee \neg v_{i, 3}) \wedge
            (\neg v_{i, 2} \vee \neg v_{i, 3})
        \end{align*}

        Weiterhin dürfen benachbarte Knoten nicht mit derselben Farbe gefärbt
        werden. Es gilt also für alle $i \in \{1, \ldots, n\}$ und
        $v_j \in N(v_i)$:
        \begin{align*}
          N = \bigwedge\limits
            (\neg v_{i, 1} \vee \neg v_{j, 1}) \wedge
            (\neg v_{i, 2} \vee \neg v_{j, 2}) \wedge
            (\neg v_{i, 3} \vee \neg v_{j, 3})
        \end{align*}

        Alle Klauseln enthalten höchstens drei Literale und die Formel $K \wedge
        N$ befindet sich in konjunktiver Normalform.

      \item Während das Halteproblem nicht entscheidbar ist, weil die
        Turingmaschine unendlich lange laufen würde, ist \COLORDrei{}
        entscheidbar, da man „nur“ endlich viele Möglichkeiten ausprobieren
        muss.
    \end{enumerate}

  \item ---

  \item
    \begin{enumerate}[(a)]
      \item Um zu zeigen, dass $\NELP{} \in NP$ ist, müssen wir zeigen, dass
        eine nichtdeterministische Turingmaschine existiert, die das Problem in
        polynomieller Zeit entscheidet.

        Angelehnt an die Vorlesung agiert die NTM nach folgendem
        Pseudoalgorithmus:
        \begin{enumerate}
          \item „rate“ ein $x \in {\{0, 1\}}^n$
          \item überprüfe $f(x_1, \ldots, x_n) \geq k$ und $Ax \leq b$
        \end{enumerate}
        Da der letzte Schritt in polynomieller Zeit ausgeführt werden kann, gilt
        die Behauptung.

      \item Gegeben sei das Problem \CLIQUE{} über dem Graphen $G = (V, E)$ mit
        $n = |V|$ und $k \in \N$. Für die Reduktion auf \NELP{} betrachten wir
        den Komplementgraph $\overline{G} = (V, V \times V \setminus E)$.

        Die Matrix $A$ ergibt sich aus der Inzidenzmatrix von $\overline{G}$:
        \begin{itemize}
          \item $A$ enthält $n$ Spalten
          \item jede Zeile von $A$ bildet eine Kante aus $\overline{G}$ ab
          \item jede Zelle von $A$ enthält eine $1$, falls der Knoten der
            jeweiligen Spalte auf der Kante liegt; ansonsten eine $0$
        \end{itemize}
        $A$ hat damit $m = \binom{n}{2} - |E|$ Zeilen und $n$ Spalten. Das $k$
        aus \CLIQUE{} ist gleich dem $k$ aus \NELP{}. $c$ ist der
        $n$-dimensionale $1$-Vektor und $b$ ist der $m$-dimensionale $1$-Vektor.

        Der Lösungsvektor $x$ gibt dann an, welche Knoten in der Clique
        enthalten sind.

      \item Die Erzeugung des Komplementgraphen benötigt maximal $\binom{n}{2} =
        \frac{n(n-1)}{2}$ Schritte und um die Matrix $A$ zu erzeugen, benötigen
        wir $n \cdot m = n \cdot \left(\frac{n(n-1)}{2} - |E|\right)$ Schritte.
        Insgesamt liegt der Aufwand in $O(n^3)$ und ist damit polynomiell.

      \item Aus
        \begin{align*}
          G = (\{1,2,3,4,5,6\}, \{12, 13, 15, 23, 25, 34, 35, 45\})
        \end{align*}
        folgt
        \begin{align*}
          \overline{G} = (\{1,2,3,4,5,6\}, \{14, 16, 24, 26, 36, 46, 56\})
        \end{align*}
        Die Matrix $A$ ist demnach
        \begin{align*}
          A & = \begin{pmatrix}
            1 & 0 & 0 & 1 & 0 & 0\\
            1 & 0 & 0 & 0 & 0 & 1\\
            0 & 1 & 0 & 1 & 0 & 0\\
            0 & 1 & 0 & 0 & 0 & 1\\
            0 & 0 & 1 & 0 & 0 & 1\\
            0 & 0 & 0 & 1 & 0 & 1\\
            0 & 0 & 0 & 0 & 1 & 1\\
          \end{pmatrix}
        \end{align*}

        Die maximale Clique hat die Größe $k = 4$ und enthält die Knoten
        $\{1,2,3,5\}$. In \NELP{} ergibt das folgende Ergebnisse:
        \begin{align*}
          Ax & = \begin{pmatrix}
            1 & 0 & 0 & 1 & 0 & 0\\
            1 & 0 & 0 & 0 & 0 & 1\\
            0 & 1 & 0 & 1 & 0 & 0\\
            0 & 1 & 0 & 0 & 0 & 1\\
            0 & 0 & 1 & 0 & 0 & 1\\
            0 & 0 & 0 & 1 & 0 & 1\\
            0 & 0 & 0 & 0 & 1 & 1\\
          \end{pmatrix}
          \begin{pmatrix}
            1\\
            1\\
            1\\
            0\\
            1\\
            0
          \end{pmatrix}
          =
          \begin{pmatrix}
            1\\
            1\\
            1\\
            1\\
            1\\
            0\\
            1
          \end{pmatrix}
          \leq
          \begin{pmatrix}
            1\\
            1\\
            1\\
            1\\
            1\\
            1\\
            1
          \end{pmatrix}
          = b
        \end{align*}
        Der optimale Zielfunktionswert ist
        \begin{align*}
          \sum\limits_{i=1}^n c_i x_i = \sum\limits_{i=1}^n x_i = 4
        \end{align*}
    \end{enumerate}
\end{enumerate}



\subsection{Serie 4}
\label{sub:serie_4}

\begin{enumerate}
  \item \begin{enumerate}
    \item Wir zeigen: die Relation $\p$ ist auf der Menge \NP{} ist eine
      Quasiordnung.

      Seien
      $L_1 \subseteq \Sigma_1^*,
        L_2 \subseteq \Sigma_2^*,
        L_3 \subseteq \Sigma_3^*$
      Entscheidungsprobleme und
      $L_1, L_2, L_3 \in \NP$.

      \begin{description}
        \item[reflexiv] Es gilt $L_1 \p L_1$ genau dann, wenn eine polynomiale
          berechenbare Funktion $f: \Sigma_1^* \rightarrow \Sigma_1^*$
          existiert, so dass für alle Wörter $w \in \Sigma_1^*$ die Äquivalenz
          $w \in L_1 \Leftrightarrow f(w) \in L_1$ gilt.

          Diese Äquivalenz ist für $f := \id$ erfüllt, also gilt $L_1 \p L_1$.

        \item[transitiv] Es ist zu zeigen: $L_1 \p L_2$ und $L_2 \p L_3$
          impliziert $L_1 \p L_3$.

          $L_1 \p L_2$ bedeutet, dass eine polynomiale berechenbare Funktion
          $f: \Sigma_1^* \rightarrow \Sigma_2^*$ existiert, so dass für alle
          Wörter
          $w \in \Sigma_1^*$ die Äquivalenz
          $w \in L_1 \Leftrightarrow f(w) \in L_2$ gilt.

          Analog dazu bedeutet $L_2 \p L_3$, dass eine polynomiale berechenbare
          Funktion
          $g: \Sigma_2^* \rightarrow \Sigma_3^*$
          existiert, so dass für alle Wörter
          $w \in \Sigma_2^*$ die Äquivalenz
          $w \in L_2 \Leftrightarrow g(w) \in L_3$ gilt.

          Damit $L_1 \p L_3$ gilt, muss eine polynomiale berechenbare Funktion
          $h: \Sigma_1^* \rightarrow \Sigma_3^*$
          existieren, so dass für alle Wörter
          $w \in \Sigma_1^*$ die Äquivalenz
          $w \in L_1 \Leftrightarrow h(w) \in L_3$ gilt.

          Wir wählen $h := g \circ f$ und erhalten
          \begin{align*}
            w \in \Sigma_1^* & \Leftrightarrow f(w) \in \Sigma_2^*\\
            f(w) \in \Sigma_2^* & \Leftrightarrow g(f(w)) \in \Sigma_3^*
              & \text{also}\\
            w \in \Sigma_1^* & \Leftrightarrow h(w) = g(f(w)) \in \Sigma_3^*
          \end{align*}
          Da $f$ und $g$ jeweils polynomiale berechenbare Funktionen sind, ist
          auch $g \circ f$ eine polynomiale berechenbare Funktion und die
          Transitivität wurde gezeigt.
      \end{description}

    \item $\p$ ist auf $\NP$ nicht antisymmetrisch.

      In der Vorlesung wurde gezeigt, dass \CLIQUE{} \NP-vollständig ist, indem
      \DreiSAT{} auf \CLIQUE{} reduziert wurde. Sicherlich existiert folgender
      Beweis: $\CLIQUE \p \DreiSAT$, da beide Probleme \NP-vollständig sind.

      Wäre $\p$ auf $\NP$ antisymmetrisch, dann wären \CLIQUE{} und \DreiSAT{}
      das gleiche Entscheidungsproblem. Dies können sie aber nicht sein, weil
      die zugrundeliegenden Sprachen (einerseits Graphen und natürliche Zahlen
      und andererseits die Sprache der Aussagenlogik) nicht gleich sind.

    \item Die Menge der \NP-vollständigen Probleme ist auf der Relation $\p$
      eine Äquivalenzrelation.
  \end{enumerate}

  \item
    \begin{enumerate}
      \item Die vorgegebene Route vom Bahnhof zum Stadion sei der gegebene
        Graph, wobei die Straßenkreuzungen die Knoten und die Straßen zwischen
        den Kreuzungen die Kanten sind.

        Mit Hilfe der Optimierungsvariante 1 von $\IS$ können wir eine minimale
        unabhängige Menge von Knoten finden, so dass alle Knoten durch diese
        Menge verbunden sind. Mit der minimalen Menge unabhängiger Knoten wird
        sichergestellt, dass das Verhältnis
        \begin{align*}
          \frac{\text{Anzahl verfügbarer Polizisten}}{\text{Anzahl zu
          überwachender Straßen}}
        \end{align*}
        maximal wird.

      \pagebreak

      \item $\TSP$ beschreibt das folgende Problem:
        \begin{description}
          \item[Eingabe:] Ein ungerichteter Graph $G$ mit ganzzahliger
            Gewichtung der Kanten und einer Grenze $k \in \N$.
          \item[Ausgabe:] „Ja“ genau dann, wenn $G$ einen Hamiltonkreis enthält,
            dass die Summe der Kantengewichte kleiner oder gleich $k$ ist.
        \end{description}

        $\HC$ beschreibt das folgende Problem:
        \begin{description}
          \item[Eingabe:] ein ungerichteter Graph $G$
          \item[Ausgabe:] „Ja“ genau dann, wenn $G$ einen Hamiltonkreis enthält.
        \end{description}

        Wir zeigen $\HC \p \TSP$.

        Zunächst müssen wir $\TSP \in \NP$ zeigen. Dazu „raten“ wir einen Pfad
        in $G$ und überprüfen, ob dieser ein Hamiltonkreis ist und ob die Summe
        der Kantengewichte kleiner oder gleich $k$ ist. Dies ist in
        polynomieller Zeit möglich.

        Sei $G$ ein ungerichteter und ungewichteter Graph, bei dem zu
        entscheiden ist, ob er einen Hamiltonkreis enthält und sei $n$ die
        Anzahl der Knoten von $G$. Wir konstruieren einen gewichteten Graphen
        $G'$, dessen Knoten und Kanten mit denen von $G$ übereinstimmen, wobei
        jede Kante ein Gewicht von $1$ besitzt. Dies ist in polynomieller Zeit
        möglich. Gegeben sei außerdem der Grenzwert $k$, der mit der Anzahl der
        Knoten $n$ in $G$ übereinstimmt.

        Dann existiert ein Hamiltonkreis mit Gewichtssumme $n$ in $G'$ genau
        dann, wenn $G$ einen Hamiltonkreis enthält.
  \end{enumerate}
\end{enumerate}



\subsection{Serie 5}
\label{sub:serie_5}

\begin{enumerate}
  \item \begin{enumerate}[1.]
      \item Gegeben: sei $S$ eine Menge und $G_1, \ldots, G_n$ Teilmengen von
        $S$. Sei $\ell \in \N$.

        Frage: Gibt es eine Auswahl von Elementen $X \subseteq S$, sodass gilt
        $|X| \leq \ell$ und
        $G_i \cap X \neq \emptyset \ (\forall i \in \{1, \ldots, n\})$.

      \item \begin{enumerate}[(a)]
        \item Zunächst betrachten wir die Definition des Problems $\VC$:
          \begin{quote}
            Gegeben sei ein Graph $G = (V,E)$ und $k \in \N$.

            Das Entscheidungsproblem von $\VC$ lautet:
            gibt es eine Menge $U \subseteq V$ mit $|U| \leq k$ und
            $e \cap U \neq \emptyset$ für alle $e \in E$?
          \end{quote}
          $f$ tut folgendes:
          \begin{itemize}
            \item $V$ aus $\VC$ wird zu $S$ aus $\IV$
            \item eine Kante $e_i$ aus $\VC$ wird zu einer Menge $G_i$ aus $\IV$
              (für alle Kanten aus $E$)
            \item $k$ aus $\VC$ wird zu $\ell$ aus $\IV$
          \end{itemize}

          Oder anders gesagt:
          \begin{align*}
            f((G,k)) = f(((V, E), k)) = \begin{cases}
              S := V\\
              G_i := e_i\\
              k := \ell
            \end{cases}
          \end{align*}

        \item Wie aus den früheren Aufgabenserien bekannt, sei
          $\Sigma = \{1, 0, \#\}$.

          Der Graph
          $G = (V, E) = (\{v_1, \ldots, v_n\}, \{e_1, \ldots, e_m\})$
          und $k \in \N$
          werden auf dem Band einer Turingmaschine wie folgt kodiert:
          \begin{align*}
            \underbrace{\bin(1) \# \bin(2) \# \ldots \# \bin(n)}_{\text{Knoten}}
            \# \#
            \underbrace{%
              \underbrace{\bin(1) \# \bin(2)}_{\text{Kante\ } (v_1, v_2)}
              \#
              \underbrace{\bin(1) \# \bin(3)}_{\text{Kante\ } (v_1, v_3)}
              \#
              \ldots
            }_{\text{Kanten}}
            \# \#
            \bin(k)
          \end{align*}

        \item \begin{description}
          \item[„$\Rightarrow$“] Sei $U \subseteq V$ eine Lösung von $\VC$,
            d.\,h. $|U| \leq k$ und $\forall e \in E: e \cap U \neq \emptyset$.

            Wir transformieren das Problem wie in (a) beschrieben und setzen
            $X := U$.

            Dann gilt $|U| = |X| \leq \ell = k$ und
            $\forall i \in \{1, \ldots, n\}: G_i \cap X \neq \emptyset$,
            da $\forall e \in E: e \cap U \neq \emptyset$ gilt.

            Damit ist eine Lösung aus $\VC$ nach der Transformation durch $f$
            auch eine Lösung aus\\ $\IV$.

          \item[„$\Leftarrow$“] Sei $X \subseteq S$ eine Lösung von
            $f((G,k)) \in \IV$, d.\,h. $|X| \leq \ell$ und
            $\forall i \in \{1, \ldots, n\}: G_i \cap X \neq \emptyset$.

            Dann ist die Lösung von $\VC$: $U := X$.

            Es gilt $|X| = |U| \leq k = \ell$ und
            $\forall e \in E: e \cap U \neq \emptyset$,
            da $\forall i \in \{1, \ldots, n\}: G_i \cap X \neq \emptyset$ gilt.

            Damit ist eine durch $f$ transformierte Lösung aus $\IV$ auch eine
            Lösung aus $\VC$.
        \end{description}

        \item Da bei der Transformation durch $f$ keinerlei Umformungen, sondern
          einfach nur Ersetzungen stattfinden, liegt der Aufwand in $O(n)$ und
          ist damit polynomiell.
      \end{enumerate}
    \end{enumerate}

    \item \begin{enumerate}
      \item Im Namen.\footnote{höhö, kleiner Scherz}

        Während beim Problem $\CLIQUE$ die größte Clique gesucht wird (und damit
        das $k$ zur Problembeschreibung gehört), ist das $k$ bei $k-\CLIQUE$
        fest gewählt.

      \item Ein Algorithmus, der einfach alle möglichen Untergraphen
        durchprobiert, hat die Komplexität $O(n^k \cdot k^2)$.

        Dies ergibt sich wie folgt: sei $n = |V|$. Von den $n$ Knoten müssen
        alle Untergraphen der Größe $k$ durchprobiert werden. Dafür gilt
        \begin{align*}
          \binom{n}{k} & = \frac{n!}{k! \cdot (n-k)!}
          = \frac{%
            \overbrace{n \cdot (n-1) \cdots (n-k+1)}^{k \text{\ Faktoren}}
            }{k!}
          \in O(n^k)
        \end{align*}
        Jeder dieser Untergraphen hat maximal $\binom{k}{2} \in O(k^2)$ Kanten.

        Insgesamt landen wir bei einer Komplexität von $O(n^k \cdot k^2)$.
      \end{enumerate}

    \item
      \begin{enumerate}
        \item Greedy-Algorithmus für $\IS$:
          \begin{algorithmic}[1]
            \REQUIRE{Graph $G = (V,E)$}
            \STATE{$U \leftarrow \emptyset$}
            \STATE{$k \leftarrow 0$}
            \WHILE{$V \neq \emptyset$}
              \STATE{$k \leftarrow$ Maximum der Anzahl ausgehender Kanten der Knoten aus $V$}
              \STATE{$v \leftarrow$ ein zufälliger Knoten aus $V$, der $k$ ausgehende Kanten hat}
              \STATE{$U \leftarrow U \cup \{ v \}$}
              \STATE{entferne $v$ und alle benachbarten Knoten von $v$ aus $V$}
            \ENDWHILE%
            \RETURN{eine unabhängige Menge $U$ von $G$}
          \end{algorithmic}
          Die Menge $U$ ist unabhängig, weil in Zeile 7 für jeden hinzugefügten
          Knoten $v$ alle benachbarten Knoten von $v$ aus $V$ entfernt werden.
          Damit können keine zwei Knoten aus $U$ benachbart sein, sodass $U$
          eine unabhängige Menge sein muss.

        \item Seien $n = |V|$ und $k = |E|$.

          Im schlimmsten Fall (nämlich dann, wenn $G$ schon eine unabhängige
          Menge ist) wird in der while-Schleife immer nur ein einziger Knoten
          aus $V$ entfernt. Die Schleife würde dann $n$-mal laufen.

          Um das Maximum der Anzahl ausgehender Kanten aller Knoten aus $V$ zu
          bestimmen, muss diese Anzahl für jeden Knoten bestimmt werden (liegt in
          $O(n \cdot k)$) und dabei das Maximum gespeichert werden. Letzteres
          können wir schon bei der Bestimmung erledigen, sodass die Komplexität
          nicht größer wird.

          Die Bestimmung von $v$ in Zeile 5 liegt in $O(n)$: wir wählen einfach
          zufällig einen Knoten aus $V$ aus und überprüfen, ob dieser $k$
          ausgehende Kanten hat. Wenn nicht, versuchen wir es nochmal.

          $U \leftarrow U \cup \{ v \}$ liegt in $O(1)$.

          Das Entfernen von $v$ und dessen Nachbarschaft aus $V$ liegt im
          schlimmsten Fall (nämlich wenn $v$ mit allen anderen Knoten verbunden
          ist) in $O(n)$.

          Insgesamt liegt die Komplexität des Algorithmus’ in
          $O(n \cdot n \cdot k) = O(n^2 \cdot k)$.

        \item\ \\
          \begin{center}
            \begin{tikzpicture}[scale=1, auto]
              \node[vertex] (d) at (0,0) {$d$};
              \node[vertex] (c) at (6,0) {$c$};
              \node[vertex] (b) at (6,6) {$b$};
              \node[vertex] (a) at (0,6) {$a$};

              \node[vertex] (e) at (2,4) {$e$};
              \node[vertex] (f) at (4,4) {$f$};
              \node[vertex] (g) at (4,2) {$g$};
              \node[vertex] (h) at (2,2) {$h$};

              \path[edge] (a) -- (b) -- (c) -- (d) -- (a);
              \path[edge] (a) -- (e) -- (h) -- (g) -- (c);
              \path[edge] (e) -- (f) -- (b);
              \path[edge] (f) -- (g);
            \end{tikzpicture}
          \end{center}

        \pagebreak
        \item 1. Lauf:
          \begin{description}
            \item[Anfang:] $k \leftarrow 0$, $U \leftarrow \emptyset$
              \begin{center}
                \begin{tikzpicture}[scale=1, auto]
                  \node[vertex] (d) at (0,0) {$d$};
                  \node[vertex] (c) at (6,0) {$c$};
                  \node[vertex] (b) at (6,6) {$b$};
                  \node[vertex] (a) at (0,6) {$a$};

                  \node[vertex] (e) at (2,4) {$e$};
                  \node[vertex] (f) at (4,4) {$f$};
                  \node[vertex] (g) at (4,2) {$g$};
                  \node[vertex] (h) at (2,2) {$h$};

                  \path[edge] (a) -- (b) -- (c) -- (d) -- (a);
                  \path[edge] (a) -- (e) -- (h) -- (g) -- (c);
                  \path[edge] (e) -- (f) -- (b);
                  \path[edge] (f) -- (g);
                \end{tikzpicture}
              \end{center}
            \item[Bestimme $k$:] $k \leftarrow 3$
            \item[Wahl von $v$:] $v \leftarrow a$,
              damit ist $U = \{ a \}$
            \item[Entferne $v$ und Nachbarn von $v$:]\ \\
              \begin{center}
                \begin{tikzpicture}[scale=1, auto]
                  \node[vertex] (c) at (6,0) {$c$};

                  \node[vertex] (f) at (4,4) {$f$};
                  \node[vertex] (g) at (4,2) {$g$};
                  \node[vertex] (h) at (2,2) {$h$};

                  \path[edge] (h) -- (g) -- (c);
                  \path[edge] (f) -- (g);
                \end{tikzpicture}
              \end{center}
            \item[Bestimme $k$:] $k \leftarrow 3$
            \item[Wahl von $v$:] $v \leftarrow g$,
              damit ist $U = \{ a, g \}$
            \item[Entferne $v$ und Nachbarn von $v$:] wir erhalten
              $V = \emptyset$

            \item[Ergebnis: ] Es wird $U = \{ a, g \}$ zurückgegeben, was eine
              optimale Lösung darstellt.
          \end{description}

          \pagebreak
          2. Lauf:
          \begin{description}
            \item[Anfang:] $k \leftarrow 0$, $U \leftarrow \emptyset$
              \begin{center}
                \begin{tikzpicture}[scale=1, auto]
                  \node[vertex] (d) at (0,0) {$d$};
                  \node[vertex] (c) at (6,0) {$c$};
                  \node[vertex] (b) at (6,6) {$b$};
                  \node[vertex] (a) at (0,6) {$a$};

                  \node[vertex] (e) at (2,4) {$e$};
                  \node[vertex] (f) at (4,4) {$f$};
                  \node[vertex] (g) at (4,2) {$g$};
                  \node[vertex] (h) at (2,2) {$h$};

                  \path[edge] (a) -- (b) -- (c) -- (d) -- (a);
                  \path[edge] (a) -- (e) -- (h) -- (g) -- (c);
                  \path[edge] (e) -- (f) -- (b);
                  \path[edge] (f) -- (g);
                \end{tikzpicture}
              \end{center}
            \item[Bestimme $k$:] $k \leftarrow 3$
            \item[Wahl von $v$:] $v \leftarrow b$,
              damit ist $U = \{ b \}$
            \item[Entferne $v$ und Nachbarn von $v$:]\ \\
              \begin{center}
                \begin{tikzpicture}[scale=1, auto]
                  \node[vertex] (d) at (0,0) {$d$};

                  \node[vertex] (e) at (2,4) {$e$};
                  \node[vertex] (g) at (4,2) {$g$};
                  \node[vertex] (h) at (2,2) {$h$};

                  \path[edge] (e) -- (h) -- (g);
                \end{tikzpicture}
              \end{center}
            \item[Bestimme $k$:] $k \leftarrow 2$
            \item[Wahl von $v$:] $v \leftarrow h$,
              damit ist $U = \{ b, h \}$
            \item[Entferne $v$ und Nachbarn von $v$:] wir erhalten
              \begin{center}
                \begin{tikzpicture}[scale=1, auto]
                  \node[vertex] (d) at (0,0) {$d$};
                \end{tikzpicture}
              \end{center}
            \item[Bestimme $k$:] $k \leftarrow 0$
            \item[Wahl von $v$:] $v \leftarrow d$,
              damit ist $U = \{ b, h, d \}$
            \item[Entferne $v$ und Nachbarn von $v$:] wir erhalten
              $V = \emptyset$

            \item[Ergebnis: ] Es wird $U = \{ b, h, d \}$ zurückgegeben, was
              keine optimale Lösung darstellt.
          \end{description}

          Bei diesen zwei Läufen sehen wir, dass eine optimale Lösung gefunden
          werden kann, dies aber nicht immer passiert.
      \end{enumerate}
\end{enumerate}



\subsection{Serie 6}
\label{sub:serie_6}

\begin{enumerate}[1.]
  \item
    \begin{enumerate}[1.]
      \item Sei
        $x = \begin{pmatrix} f_H\\ f_B\\ \ell_H\\ \ell_B \end{pmatrix} \in {\{0, 1\}}^4$
        mit
        \begin{description}
          \item[$f_H$] Fabrikbau in Hamburg
          \item[$f_B$] Fabrikbau in Berlin
          \item[$\ell_H$] Lagerhalle in Hamburg
          \item[$\ell_B$] Lagerhalle in Berlin
        \end{description}
        der gesuchte Lösungsvektor. Die Nebenbedingungen ergeben sich aus
        folgender Ungleichung:
        \begin{align*}
          \begin{pmatrix}
            1  & 1  &    & \\
               &    & 1  & 1\\
            -1 & -1 &    & \\
               & -1 &    & 1\\
            -1 &    & 1  & \\
            1  &    &    & \\
            -1 &    &    & \\
               & 1  &    & \\
               & -1 &    & \\
               &    & 1  & \\
               &    & -1 & \\
               &    &    & 1\\
               &    &    & -1\\
            6  & 3  & 5  & 2
          \end{pmatrix}
          \begin{pmatrix}
            f_H\\
            f_B\\
            \ell_H\\
            \ell_B
          \end{pmatrix}
          \leq
          \begin{pmatrix}
            2\\
            1\\
            -1\\
            0\\
            0\\
            1\\
            0\\
            1\\
            0\\
            1\\
            0\\
            1\\
            0\\
            10
          \end{pmatrix}
        \end{align*}
        Die zu maximierende Zielfunktion lautet
        \begin{align*}
          f(f_H, f_B, \ell_H, \ell_B) = 9 f_H + 5 f_B + 6 \ell_H + 4 \ell_B
        \end{align*}
        also ist $c = \begin{pmatrix}
          9\\
          5\\
          6\\
          4
        \end{pmatrix}$.

      \pagebreak

      \item Die Informationen eines Knotens im Ableitungsbaum seien wie folgt
        kodiert:
        \begin{center}
          \begin{tikzpicture}[scale=1]
            \node [draw] {%
              \begin{tabular}{l|c|c}
                $P_i$
                & $\overline{F_i}$
                & $\underline{F_i}$\\
                \midrule
                \multicolumn{3}{c}{%
                  $v_i$
                }
              \end{tabular}
            };
          \end{tikzpicture}
        \end{center}
        Damit ergibt sich folgender Ableitungsbaum:
        \begin{center}
          \begin{tikzpicture}[
              scale=1,
              level distance=35mm,
              level 1/.style={sibling distance=45mm},
              level 2/.style={sibling distance=80mm},
              level 3/.style={sibling distance=40mm},
              every child node/.style={draw}
          ]
            \node [draw] {%
              \begin{tabular}{l|c|c}
                $P_1$
                & $16{,}5$
                & $0$\\
                \midrule
                \multicolumn{3}{c}{%
                  $\begin{pmatrix}
                    0{,}8 & 1 & 0 & 1
                  \end{pmatrix}$
                }
              \end{tabular}
            }
              child {%
                node {%
                  \begin{tabular}{l|c|c}
                    $P_2$
                    & $9$
                    & $9$\\
                    \midrule
                    \multicolumn{3}{c}{%
                      $\begin{pmatrix}
                        0 & 1 & 0 & 1
                      \end{pmatrix}$
                    }
                  \end{tabular}
                }
                edge from parent node [near end, above left] {$f_H = 0$}
              }
              child {%
                node {%
                  \begin{tabular}{l|c|c}
                    $P_3$
                    & $16{,}2$
                    & $9$\\
                    \midrule
                    \multicolumn{3}{c}{%
                      $\begin{pmatrix}
                        1 & 0{,}8 & 0 & 0{,}8
                      \end{pmatrix}$
                    }
                  \end{tabular}
                }
                child {%
                  node {%
                    \begin{tabular}{l|c|c}
                      $P_4$
                      & $13{,}8$
                      & $9$\\
                      \midrule
                      \multicolumn{3}{c}{%
                        $\begin{pmatrix}
                          1 & 0 & 0{,}8 & 0
                        \end{pmatrix}$
                      }
                    \end{tabular}
                  }
                  child {%
                    node {%
                      \begin{tabular}{l|c|c}
                        $P_5$
                        & $9$
                        & $9$\\
                        \midrule
                        \multicolumn{3}{c}{%
                          $\begin{pmatrix}
                            1 & 0 & 0 & 0
                          \end{pmatrix}$
                        }
                      \end{tabular}
                    }
                    edge from parent node [near end, above left] {$\ell_H = 0$}
                  }
                  child {%
                    node {%
                      \begin{tabular}{l|c|c}
                        $P_6$ & \multicolumn{2}{c}{$\emptyset$}\\
                        \midrule
                        \multicolumn{3}{c}{%
                          Keine opt. Lösung.
                        }
                      \end{tabular}
                    }
                    edge from parent node [near end, above right] {$\ell_H = 1$}
                  }
                  edge from parent node [near end, above left] {$f_B = 0$}
                }
                child {%
                  node {%
                    \begin{tabular}{l|c|c}
                      $P_7$
                      & $16$
                      & $9$\\
                      \midrule
                      \multicolumn{3}{c}{%
                        $\begin{pmatrix}
                          1 & 1 & 0 & 0{,}5
                        \end{pmatrix}$
                      }
                    \end{tabular}
                  }
                  child {%
                    node {%
                      \begin{tabular}{l|c|c}
                        $P_8$
                        & $15{,}2$
                        & $9$\\
                        \midrule
                        \multicolumn{3}{c}{%
                          $\begin{pmatrix}
                            1 & 1 & 0{,}2 & 0
                          \end{pmatrix}$
                        }
                      \end{tabular}
                    }
                    child {%
                      node {%
                        \begin{tabular}{l|c|c}
                          $P_9$
                          & $14$
                          & $9$\\
                          \midrule
                          \multicolumn{3}{c}{%
                            $\begin{pmatrix}
                              1 & 1 & 0 & 0
                            \end{pmatrix}$
                          }
                        \end{tabular}
                      }
                      edge from parent node [near end, above left] {$\ell_H = 0$}
                    }
                    child {%
                      node {%
                        \begin{tabular}{l|c|c}
                          $P_{10}$ & \multicolumn{2}{c}{$\emptyset$}\\
                          \midrule
                          \multicolumn{3}{c}{%
                            Keine opt. Lösung.
                          }
                        \end{tabular}
                      }
                      edge from parent node [near end, above right] {$\ell_H = 1$}
                    }
                    edge from parent node [near end, above left] {$\ell_B = 0$}
                  }
                  child {%
                    node {%
                      \begin{tabular}{l|c|c}
                        $P_{11}$ & \multicolumn{2}{c}{$\emptyset$}\\
                        \midrule
                        \multicolumn{3}{c}{%
                          Keine opt. Lösung.
                        }
                      \end{tabular}
                    }
                    edge from parent node [near end, above right] {$\ell_B = 1$}
                  }
                  edge from parent node [near end, above right] {$f_B = 1$}
                }
                edge from parent node [near end, above right] {$f_H = 1$}
              };
          \end{tikzpicture}
        \end{center}
        Bei $P_5$ ist die Suche ausgelotet, weil $\overline{F_5} =
        \underline{F_5}$ gilt. Bei $P_6, P_{10}$ und $P_{11}$ ist die Suche
        ausgelotet, weil keine optimale Lösung gefunden wurde.

        Die optimale Lösung ist daher $P_9$, das heißt, dass Fabriken in Hamburg
        und Berlin gebaut werden, auf Lagerhallen jedoch verzichtet wird.

    \end{enumerate}

  \pagebreak

  \item \begin{enumerate}
    \item
      \begin{align*}
        S & \stackrel{(1)}{\rightarrow}
          \texttt{if\textvisiblespace{}condition\textvisiblespace{}then\textvisiblespace{}}SE\\
          & \stackrel{(4)}{\rightarrow}
          \texttt{if\textvisiblespace{}condition\textvisiblespace{}then\textvisiblespace{}}S\texttt{\textvisiblespace{}else\textvisiblespace{}}S\\
          & \stackrel{(1)}{\rightarrow}
          \texttt{if\textvisiblespace{}condition\textvisiblespace{}then\textvisiblespace{}}\texttt{if\textvisiblespace{}condition\textvisiblespace{}then\textvisiblespace{}}SE\texttt{\textvisiblespace{}else\textvisiblespace{}}S\\
          & \stackrel{(2)}{\rightarrow}
          \texttt{if\textvisiblespace{}condition\textvisiblespace{}then\textvisiblespace{}}\texttt{if\textvisiblespace{}condition\textvisiblespace{}then\textvisiblespace{}statement;}E\texttt{\textvisiblespace{}else\textvisiblespace{}}S\\
          & \stackrel{(3)}{\rightarrow}
          \texttt{if\textvisiblespace{}condition\textvisiblespace{}then\textvisiblespace{}}\texttt{if\textvisiblespace{}condition\textvisiblespace{}then\textvisiblespace{}statement;}\texttt{\textvisiblespace{}else\textvisiblespace{}}S\\
          & \stackrel{(2)}{\rightarrow}
          \texttt{if\textvisiblespace{}condition\textvisiblespace{}then\textvisiblespace{}}\texttt{if\textvisiblespace{}condition\textvisiblespace{}then\textvisiblespace{}statement;}\texttt{\textvisiblespace{}else\textvisiblespace{}statement;}
      \end{align*}

    \item Die Ableitung eines Wortes ist nicht immer eindeutig. Das Wort
      \begin{center}
        \texttt{if\textvisiblespace{}condition\textvisiblespace{}then\textvisiblespace{}if\textvisiblespace{}condition\textvisiblespace{}then\textvisiblespace{}statement;\textvisiblespace{}else\textvisiblespace{}statement;}
      \end{center}
      hat zwei verschiedene Ableitungsbäume. Einerseits:
      \begin{center}
        \begin{tikzpicture}[
            scale=1,
            level distance=15mm,
            level 1/.style={sibling distance=50mm},
            level 2/.style={sibling distance=25mm},
            level 3/.style={sibling distance=40mm},
          ]
          \node {$S$}
            child {node {\texttt{if\textvisiblespace{}condition\textvisiblespace{}then\textvisiblespace{}}}}
            child {%
              node {$S$}
              child {node {\texttt{if\textvisiblespace{}condition\textvisiblespace{}then\textvisiblespace{}}}}
              child {%
                node {$S$}
                child {node {\texttt{statement;}}}
              }
              child {%
                node {$E$}
                child
              }
            }
            child {%
              node {$E$}
              child {node {\texttt{\textvisiblespace{}else\textvisiblespace{}}}}
              child {%
                node {$S$}
                child {node {\texttt{statement;}}}
              }
            }
          ;
        \end{tikzpicture}
      \end{center}
      Andererseits:
      \begin{center}
        \begin{tikzpicture}[
            scale=1,
            level distance=15mm,
            level 1/.style={sibling distance=50mm},
            level 2/.style={sibling distance=25mm},
            level 3/.style={sibling distance=10mm},
            level 4/.style={sibling distance=10mm},
          ]
          \node {$S$}
            child {node {\texttt{if\textvisiblespace{}condition\textvisiblespace{}then\textvisiblespace{}}}}
            child {%
              node {$S$}
              child {node {\texttt{if\textvisiblespace{}condition\textvisiblespace{}then\textvisiblespace{}}}}
              child {%
                node {$S$}
                child {node {\texttt{statement;}}}
              }
              child {%
                node {$E$}
                child {node {\texttt{\textvisiblespace{}else\textvisiblespace{}}}}
                child {%
                  node {$S$}
                  child {node {\texttt{statement;}}}
                }
              }
            }
            child {%
              node {$E$}
              child
            }
          ;
        \end{tikzpicture}
      \end{center}
      Damit ist die Grammatik mehrdeutig. Das liegt daran, dass die rechte Seite
      der Regel $E \rightarrow \varepsilon$ ein Präfix der rechten Seite der
      Regel $E \rightarrow \texttt{\textvisiblespace{}else\textvisiblespace{}}S$
      ist.
  \end{enumerate}

  \pagebreak

  \item
    \begin{enumerate}[(a)]
      \item Sei $G = (X,H,R,S)$ mit
        \begin{align*}
          X & = \{a, b, c\}\\
          H & = \{S, A, B, C\}\\
          R & = \{
            (S, \varepsilon), (S, ABC),\\
            & \phantom{=}\quad (AB, BA), (BA, AB),\\
            & \phantom{=}\quad (AC, CA), (CA, AC),\\
            & \phantom{=}\quad (BC, CB), (CB, BC),\\
            & \phantom{=}\quad (A, AABC), (B, BABC), (C, CABC),\\
            & \phantom{=}\quad (A, a), (B, b), (C, c)\}
        \end{align*}

      \item
        \begin{align*}
          S & \rightarrow ABC \rightarrow ACB \rightarrow ACABCB\\
          & \rightarrow aCABCB \rightarrow acABCB \rightarrow acaBCB
            \rightarrow acabCB \rightarrow acabcB \rightarrow acabcb
        \end{align*}

      \item Die Grammatik ist längenmonoton, weil einerseits für alle
        $(l, r) \in R$ mit $l \neq S$ die Eigenschaft $|l| \leq |r|$
        gilt, und andererseits, da $(S, \varepsilon) \in R$, das
        Nichtterminal $S$ auf keiner rechten Seite in $R$ vorkommt.
    \end{enumerate}
\end{enumerate}



\subsection{Serie 7}
\label{sub:serie_7}

\begin{enumerate}[1.]
  \item
    \begin{enumerate}[(a)]
      \item Wir wählen $n = 1$. Damit gilt $x = x_0 R x_1 = y$, also sind alle
        Tupel $(x,y)$ aus $R$ auch in $R^+$ ($R \subseteq R^+$).

      \item Wir starten mit $n = 2$:
        \begin{itemize}
          \item $x = aRbRc = y \Rightarrow (a,c) \in R^+$
          \item $x = aRbRe = y \Rightarrow (a,e) \in R^+$
          \item $x = bReRd = y \Rightarrow (b,d) \in R^+$
        \end{itemize}

        Weiter geht’s mit $n = 3$:
        \begin{itemize}
          \item $x = aRbReRd = y \Rightarrow (a,d) \in R^+$
        \end{itemize}

        Und das war’s auch schon. Damit ist
        $R^+ = \{(a,b), (b,c), (b,e), (e,d), (a,c), (a,e), (b,d), (a,d)\}$

      \item $R^* = R^+ \cup \{(a,a), (b,b), (c,c), (d,d), (e,e)\}$
    \end{enumerate}

  \item
    \begin{enumerate}[(a)]
      \item Damit $E \subseteq R$ gilt, müssen wir für alle $(u,v) \in E$ je
        ein $w_1$ und $w_2$ finden, sodass $w = w_1 u w_2$ und $\tilde{w} = w_1
        v w_2$ und damit $w R \tilde{w}$ gilt.

        Dafür wählen wir $w_1 = w_2 = \epsilon$ und damit ist $w = u$ und
        $\tilde{w} = v$. Da $(u,v) \in E$, gilt $wR\tilde{w}$.

      \item\
        \begin{center}
          \begin{tabular}{llllll}
            \toprule
            $w$ & $\tilde{w}$ & $w_1$ & $u$ & $v$& $w_2$\\
            \midrule
            $aX$ & $acZ$ & $a$ & $X$ & $cZ$ & $\epsilon$\\
            $acZ$ & $aca$ & $ac$ & $Z$ & $a$ & $\epsilon$\\
            $bY$ & $bdU$ & $b$ & $Y$ & $dU$ & $\epsilon$\\
            $bdU$ & $bdb$ & $bd$ & $U$ & $b$ & $\epsilon$\\
            $cZ$ & $ca$ & $c$ & $Z$ & $a$ & $\epsilon$\\
            $dU$ & $db$ & $d$ & $U$ & $b$ & $\epsilon$
          \end{tabular}
        \end{center}

      \item $\{ (S, w) \in \{ S \} \times {(X \cup H)}^* \mid S R^+ w\}
        = \{ (S, aX), (S, acZ), (S, aca), (S, bY), (S, bdU), (S, bdb) \}$

      \item $\{ (S, w) \in \{ S \} \times {(X \cup H)}^* \mid S R^* w\}
        = \{ (S, w) \in \{ S \} \times {(X \cup H)}^* \mid S R^+ w\}
        \cup \{ (S,S) \}$
      \item $\{ aca, bdb \}$
    \end{enumerate}

  \item
    \begin{enumerate}[(a)]
      \item Sei $G = (X, H, E, S)$ mit
        \begin{align*}
          X & = \{a, b, c\}\\
          H & = \{ S \}\\
          E & = \{
            (S, \epsilon),
          (S, a), (S, b), (S, c),
          (S, aSa), (S, bSb), (S, cSc)
          \}
        \end{align*}

      \item Ich nehme mal an, dass wir folgende Aussage beweisen
        sollen:\footnote{Sonst wäre diese Aufgabe die gleiche wie 3 (c).}
        \begin{center}
          Wenn $w \in {\{a, b, c\}}^*$ ein Palindrom ist, dann ist $w \in L_G$.
        \end{center}
        Sei $\ell: X^* \rightarrow \N$ die Wortlängenfunktion.

        Induktionsanfang: sei $w \in {\{a, b, c\}}^*$ und $\ell(w) \leq 1$.
        Ausgehend vom Startsymbol $S$ wurde dann eine der folgenden
        Ableitungsregeln angewandt: $(S, \varepsilon), (S, a), (S, b), (S, c)$.
        Damit ist $w \in L_G$.

        Induktionsvoraussetzung: sei $w \in {\{a, b, c\}}^*$ und $\ell(w) = n$,
        dann ist $w \in L_G$.

        Induktionsschritt: sei $v = v_1 v_2 \ldots v_{n+2} \in {\{a, b, c\}}^*$
        ein Palindrom mit $\ell(v) = n+2$. Dann gilt nach Definition von
        Palindromen:
        \begin{itemize}
          \item Es existiert ein Palindrom $w \in {\{a, b, c\}}^*$ mit $w = v_2
            \ldots v_{n+1}$ und $\ell(w) = n$, welches nach
            Induktionsvoraussetzung in $L_G$ liegt.

          \item $v_1 = v_{n+2}$ und damit wurde in der Ableitung von $v$ eine
            der folgenden Regeln benutzt:
            \begin{align*}
              (S, aSa), (S, bSb), (S, cSc)
            \end{align*}
        \end{itemize}
        Damit gilt $v \in L_G$.

      \item Induktionsanfang: Sei $w \in L_G$ und $\ell(w) = 0$, also $w =
        \varepsilon$, dann ist $w$ nach Definition von Palindromen
        (Induktionsanfang) ein Palindrom.

        Sei $v \in L_G$ und $\ell(v) = 1$, dann ist auch $v$ nach Definition von
        Palindromen (Induktionsanfang) ein Palindrom.

        Induktionsvoraussetzung: sei $w \in L_G$ und $\ell(w) = n$, dann ist $w$
        ein Palindrom.

        Induktionsschritt: sei $v = v_1 v_2 \ldots v_{n+2} \in L_G$ mit $\ell(v)
        = n+2$. Da $v \in L_G$ und $\ell(v) \geq 2$ ist, muss bei der Ableitung
        von $v$ eine der folgenden Ableitungsregeln angewandt worden sein:
        \begin{align*}
          (S, aSa), (S, bSb), (S, cSc)
        \end{align*}
        Daraus folgt $v_1 = v_{n+2}$.

        Da $v \in L_G$, ist das Teilwort $w = v_2 \ldots v_{n+1}$ auch in $L_G$.
        Weiterhin gilt $\ell(w) = n$, daher ist $w$ nach Induktionsvoraussetzung
        ein Palindrom.

        Insgesamt erhalten wir: $v$ ist ein Palindrom.
    \end{enumerate}

  \item
    \begin{enumerate}[(a)]
      \item Sei $G = (X, H, E, S)$ mit
        \begin{align*}
          X & = \{a, b\}\\
          H & = \{S, B\}\\
          E & = \{(S, \varepsilon), (S, abB), (S, B), (B, \varepsilon), (B, bbB)\}
        \end{align*}

      \item $S \rightarrow abB \rightarrow abbbB \rightarrow abbbbbB \rightarrow
        abbbbb$ und $S \rightarrow B \rightarrow bbB \rightarrow bbbbB
        \rightarrow bbbbbbB \rightarrow bbbbbb$.

      \item Sei $A = (X, Q, q_0, \delta, F)$ der DEA, der $L$ akzeptiert. Es gilt:
        \begin{align*}
          X & = \{a, b\}\\
          Q & = \{q_0, q_1, q_2, q_3, q_4, q_5\}\\
          F & = \{q_0, q_2\}
        \end{align*}
        Weiterhin ist $\delta: Q \times X \rightarrow Q$ wie folgt definiert:
        \begin{center}
          \begin{tabular}{lcc}
            \toprule
                  & $a$ & $b$\\
            \midrule
            $q_0$ & $q_1$ & $q_3$\\
            $q_1$ & $q_4$ & $q_2$\\
            $q_2$ & $q_4$ & $q_3$\\
            $q_3$ & $q_4$ & $q_2$\\
            $q_4$ & $q_4$ & $q_4$\\
            \bottomrule
          \end{tabular}
        \end{center}
  \end{enumerate}
\end{enumerate}


\subsection{Serie 10}
\label{sub:serie_10}

\begin{enumerate}[1.]
  \item
    \begin{enumerate}
      \item A1:
        \begin{align*}
          \begin{matrix}
              & a      & b      & c      & d      & e      & f\\
            a & \times &        & \times & \times & \times & \times\\
            b &        & \times & \times & \times & \times & \times\\
            c & \times & \times & \times &        & \times & \times\\
            d & \times & \times &        & \times & \times & \times\\
            e & \times & \times & \times & \times & \times & \\
            f & \times & \times & \times & \times &        & \times
          \end{matrix}
        \end{align*}
        \begin{align*}
          [a] & = \{ a, b \}\\
          [b] & = \{ a, b \}\\
          [c] & = \{ c, d \}\\
          [d] & = \{ c, d \}\\
          [e] & = \{ e, f \}\\
          [f] & = \{ e, f \}
        \end{align*}

      \item Der Myhill-Nerode-Index beträgt $3$, da A1 genau $3$
        Äquivalenzklassen besitzt ($\{a,b\}, \{c,d\}, \{e,f\}$).

      \item \
        \begin{center}
          \begin{tikzpicture}[scale=1,node distance=0.8cm,auto]
            \node [state, initial] (ab) {$a,b$};
            \node [state, accepting] (cd) [below=of ab] {$c,d$};
            \node [state] (ef) [below=of cd] {$e,f$};

            \path[->] (ab)  edge [loop right] node {$0$}    ();
            \path[->] (ab)  edge              node {$1$}    (cd)
                      (cd)  edge              node {$0,1$}  (ef);
            \path[->] (ef)  edge [loop right] node {$0,1$}  ();
          \end{tikzpicture}
        \end{center}

      \item A2:
        \begin{align*}
          \begin{matrix}
                 & z_0    & z_1    & z_2    & z_3    & z_4    & z_5    & z_6    & z_7\\
            z_0  & \times & \times & \times & \times &        &        & \times & \times\\
            z_1  & \times & \times & \times & \times & \times & \times & \times & \times\\
            z_2  & \times & \times & \times & \times & \times & \times & \times & \\
            z_3  & \times & \times & \times & \times & \times & \times & \times & \times\\
            z_4  &        & \times & \times & \times & \times &        & \times & \times\\
            z_5  &        & \times & \times & \times &        & \times & \times & \times\\
            z_6  & \times & \times & \times & \times & \times & \times & \times & \times\\
            z_7  & \times & \times &        & \times & \times & \times & \times & \times\\
          \end{matrix}
        \end{align*}
        \begin{align*}
          [z_0] & = \{z_0, z_4, z_5\}\\
          [z_1] & = \{z_1\}\\
          [z_2] & = \{z_2, z_7\}\\
          [z_3] & = \{z_3\}\\
          [z_4] & = \{z_0, z_4, z_5\}\\
          [z_5] & = \{z_0, z_4, z_5\}\\
          [z_6] & = \{z_6\}\\
          [z_7] & = \{z_2, z_7\}
        \end{align*}

      \item A2 hat den Myhill-Nerode-Index $5$, weil es genau $5$
        Äquivalenzklassen gibt:
        \begin{align*}
          \{z_0, z_4, z_5\}, \{z_1\}, \{z_2, z_7\}, \{z_3\}, \{z_6\}
        \end{align*}

      \item \
        \begin{center}
        \begin{tikzpicture}[scale=1,node distance=0.8cm,auto]
          \node [state, initial] (045) {$z_0, z_4, z_5$};
          \node [state] (1) [below=of 045] {$z_1$};
          \node [state, accepting] (27) [below=of 1] {$z_2, z_7$};
          \node [state] (3) [below right=of 27] {$z_3$};
          \node [state] (6) [below left=of 27] {$z_6$};

          \path[->] (045) edge [loop right] node        {$1$}   ();
          \path[->] (045) edge [bend left]  node        {$0$}   (1);
          \path[->] (1)   edge [bend left]  node        {$0$}   (045)
                          edge              node        {$1$}   (27);
          \path[->] (27)  edge              node        {$0$}   (3)
                          edge [bend left]  node        {$1$}   (6);
          \path[->] (6)   edge [bend left]  node        {$1$}   (27)
                          edge [bend right] node [swap] {$0$}   (3);
          \path[->] (3)   edge [loop right] node        {$0,1$} ();
        \end{tikzpicture}
        \end{center}

      \item Ein DEA ohne totale Zustandsübergangsfunktion hat Zustände, für die
        der Übergang bei bestimmten Zeichen nicht definiert ist. Das bedeutet,
        dass keine Zustandsänderung stattfindet und der DEA in diesem Zustand
        bleibt.

        Das heißt, dass man einen DEA ohne totale Zustandsübergangsfunktion zu
        einen mit totaler Zustandsübergangsfunktion erweitern kann, indem man
        zusätzliche Schleifen hinzufügt.

        Diesen DEA kann man dann ganz normal minimieren.
    \end{enumerate}

  \item
    \begin{enumerate}
      \item \
        \begin{center}
        \begin{tikzpicture}[scale=1, node distance=1cm, auto]
          \node [state, initial] (q_0) {$q_0$};
          \node [state] (q_1) [below right=of q_0] {$q_1$};
          \node [state] (q_3) [below left=of q_0] {$q_3$};
          \node [state, accepting] (q_2) [below left=of q_1] {$q_2$};

          \path[->] (q_0) edge node {$a$} (q_1)
                          edge node [swap] {$b$} (q_3)
                    (q_1) edge node {$a$} (q_3)
                          edge node {$b$} (q_2)
                    (q_3) edge [loop left] node {$a,b$} ()
                    (q_2) edge [loop below] node {$a,b$} ();
        \end{tikzpicture}
        \end{center}
      \item $L(A) = \{abw \mid w \in \{a, b\}^*\}$

      \item Da sich der Automat nicht weiter minimieren lässt, ist die minimale
        Pumpingzahl der Sprache gleich der Anzahl der Zustände des Automats und
        damit $4$.
      
    \end{enumerate}

\end{enumerate}


\end{document}
