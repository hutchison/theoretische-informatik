\documentclass[
  a4paper,
  11pt,
]{scrartcl}

\usepackage[utf8]{inputenc}
\usepackage[cm, headings]{fullpage}
\usepackage[ngerman]{babel}
\usepackage{amsmath}
\usepackage{amssymb}
% Für die Klammern der Interpretations-abbildung
\usepackage{stmaryrd}

\usepackage{color}
\definecolor{mygray}{rgb}{0.5,0.5,0.5}
\usepackage{tikz}
\usepackage{pgfplots}

% Für Zeilenumbrüche ohne Indentations
\setlength{\parindent}{0pt}

\usepackage{listings}
\lstset{%
  basicstyle=\ttfamily,
  keywordstyle=\color{blue},
  commentstyle=\color{mygray},
  language=C,
  showstringspaces=false,
}

\usepackage{enumerate}

% coole Kopf- und Fußzeilen:
\usepackage{fancyhdr}
% Seitenstil ist natürlich fancy:
\pagestyle{fancy}
% alle Felder löschen:
\fancyhf{}

%\fancyhead[L]{
%}
%\fancyhead[R]
%\fancyfoot[L]{}

\fancyfoot[C]{\thepage}

\title{Klausurvorbereitung: Theoretische Informatik}

\subtitle{Universität Rostock}

%\author

\date{}

\newcommand{\p}{\leq_{\textsf{p}}}
\newcommand{\N}{\mathbb{N}}
\renewcommand{\O}{\mathcal{O}}

\newcommand{\COLOR}{\textsf{COLOR}}
\newcommand{\COLORDrei}{\textsf{COLOR-3}}
\newcommand{\DreiSAT}{\textsf{3SAT}}
\newcommand{\bin}{\text{bin}}
\newcommand{\TSP}{\textsf{TSP}}
\newcommand{\HC}{\textsf{HAMILTONIAN CIRCUIT}}
\newcommand{\TSPOPTI}{\textsf{TSP\_OPT\_1}}
\newcommand{\NELP}{\textsf{0--1 LINEAR PROGRAMMING}}
\newcommand{\CLIQUE}{\textsf{CLIQUE}}
\newcommand{\NP}{\textsf{NP}}
\newcommand{\id}{\text{id}}
\newcommand{\IS}{\textsf{INDEPENDENT SET}}
\newcommand{\VC}{\textsf{VERTEX COVER}}
\newcommand{\IV}{\textsf{INTERESSENVERTRETER}}

\begin{document}

\maketitle

\section*{Themen \& Inhalte}
\label{sec:themen}

\begin{itemize}
  \item \NP-Vollständigkeit durch Polynomialreduktion zeigen
    \begin{itemize}
      \item Bsp: $\DreiSAT \p \IS$
    \end{itemize}
  \item Zeigen, dass eine Sprache nicht regulär ist (Pumping-Lemma für reguläre
    Sprachen)
    \begin{itemize}
      \item Bsp: $L = \{ 0^n 1^n \mid n \in \N \}$ ist nicht regulär
    \end{itemize}
  \item Turing-Maschine bauen (deterministisch und nichtdeterministisch)
    \begin{itemize}
      \item Bsp: eine NTM mit $\Sigma = \{\Box, 0, 1\}$, die zwei Einsen auf dem
        Band erkennt (Startposition ist beliebig)
    \end{itemize}
\end{itemize}

\end{document}
