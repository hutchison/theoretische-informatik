\begin{enumerate}
  \item
    \begin{enumerate}[(a)]
      \item Sei $\Sigma$ ein beliebiges nichtleeres Alphabet. Es ist $L \in P$,
        genau dann wenn $L \subseteq \Sigma^*$ und eine Turingmaschine $M$
        existiert, die die char. Funktion
        $\chi_L: \Sigma^* \rightarrow \{0, 1\}$
        in polynomieller Zeit berechnet. Im Endzustand von $M$ steht auf dem
        Band demnach:
        \begin{align*}
          M(\omega) = \begin{cases}
            1 & \omega \in L\\
            0 & \omega \notin L
          \end{cases}
        \end{align*}
        Dazu betrachten wir die Turingmaschine $M'$, die im Endzustand auf dem
        Band folgendes Ergebnis liefert:
        \begin{align*}
          M'(\omega) = \begin{cases}
            1 & M(\omega) = 0\\
            0 & M(\omega) = 1
          \end{cases}
        \end{align*}
        Die Turingmaschine $M'$ berechnet also in polynomieller Zeit, ob $\omega
        \in \Sigma^* \setminus L$ ist. Daraus folgt $P \subseteq coP$.

        Analog dazu betrachten wir $L \in coP$: es ist $L \in coP$ genau dann,
        wenn $L \subseteq \Sigma^*$ und eine Turingmaschine $M$ existiert, die die
        char. Funktion
        $\chi_L: \Sigma^* \setminus L \rightarrow \{0, 1\}$
        in polynomieller Zeit berechnet. Im Endzustand von $M$ steht auf dem
        Band demnach:
        \begin{align*}
          M(\omega) = \begin{cases}
            1 & \omega \in \Sigma^* \setminus L\\
            0 & \omega \notin \Sigma^* \setminus L
          \end{cases}
        \end{align*}
        Wir definieren $M'$ wie oben und erhalten das Ergebnis, dass $M'$ in
        polynomieller Zeit berechnet, ob $\omega \in L$ ist. Daraus folgt $coP
        \subseteq P$ und insgesamt $P = coP$.

      \item Sei $L_1 \in P$ und $L_2 \in P$. Das heißt, dass die Turingmaschinen
        $M_1$ und $M_2$ existieren, die $L_1$ bzw. $L_2$ in polynomieller Zeit
        entscheiden. Der Aufwand von $M_1$ für $L_1$ sei in $O(n^{k_1})$ und der
        Aufwand von $M_2$ für $L_2$ in $O(n^{k_2})$ (für konstante Werte $k_1$
        und $k_2$).

        Die Konkatenation $L_1 L_2$ ist wie folgt definiert:
        \begin{align*}
          L_1 L_2 = \{ \omega_1 \omega_2 \mid \omega_1 \in L_1, \omega_2 \in L_2
          \}
        \end{align*}
        Wir definieren eine Turingmaschine $M$, die $L_1 L_2$ in polynomieller
        Zeit entscheidet. Diese muss ein Eingabewort $\omega$ jedoch zunächst
        so in $\omega_1$ und $\omega_2$ aufspalten, dass $\omega_1 \in L_1$ und
        $\omega_2 \in L_2$ gilt.

        Dies erfolgt nach folgendem Pseudocode:
        \begin{algorithmic}[1]
          \REQUIRE{$\omega = a_1 a_2 \ldots a_n$}
          \FOR{$i = 0$ \textbf{to} $n$}\label{alg:schleife}
          \STATE{teile $\omega$ in $\omega_1 = a_1 \ldots a_i$ und $\omega_2 = a_{i+1} \ldots a_n$}\label{alg:aufteilung}
          \STATE{berechne Ergebnis von $M_1$ mit der Eingabe $\omega_1$}
          \STATE{berechne Ergebnis von $M_2$ mit der Eingabe $\omega_2$}
          \IF{$M_1$ und $M_2$ akzeptieren ihre Eingaben}
            \STATE{akzeptiere $\omega$}
          \ENDIF{}
          \ENDFOR{}
          \IF{keine Aufteilung von $\omega$ in $\omega_1$ und $\omega_2$ führte
          in einen akzeptierenden Zustand von $M_1$ und $M_2$}
            \STATE{lehne $\omega$ ab}
          \ENDIF{}
        \end{algorithmic}
        Bei der Aufteilung von $\omega$ in $\omega_1$ und $\omega_2$ in
        Zeile~\ref{alg:aufteilung} gilt
        $a_1 \ldots a_0 = \epsilon = a_{n+1} \ldots a_n$.

        Es bleibt zu zeigen, dass $M$ die Eingabe $\omega$ in polynomieller Zeit
        akzeptiert. Wie schon erwähnt liegt der Aufwand von $M_1$ für $L_1$ in
        $O(n^{k_1})$ und der Aufwand von $M_2$ für $L_2$ in $O(n^{k_2})$.
        Insgesamt liegt der Aufwand für einen Schleifendurchlauf in $O(n^{k_1})
        + O(n^{k_2}) = O(n^k)$ mit $k = \max(k_1, k_2)$. Dazu kommt noch der
        Aufwand für die Aufteilung von $\omega$ in $\omega_1$ und $\omega_2$.
        Die Schleife in Zeile~\ref{alg:schleife} wird dabei maximal $n+1$ mal
        durchlaufen. Das heißt, dass der Aufwand für die Entscheidung von $L_1
        L_2$ gleich $(n+1) \cdot O(n^k)$ ist und damit in $O(n^{k+1})$ liegt,
        was polynomiellem Aufwand entspricht. Es gilt also $L_1 L_2 \in P$.

      \item —
    \end{enumerate}

  \item
    \begin{enumerate}[(a)]
    \item Gegeben sei ein ungerichteter Graph $(V,E)$.

      Finde das minimale $k$, sodass man die Knoten mit $k$ Farben so färben
        kann, dass benachbarte Knoten immer verschiedene Farben haben?

    \item Gegeben sei ein ungerichteter Graph $(V,E)$ mit $V = \{v_1, \ldots,
        v_n \}$. Wir können den Graphen mit maximal $n$ Farben färben und falls
        der Graph nur einen Knoten besitzt, können wir ihn mit genau einer Farbe
        färben. Der minimale Wert für $k$ befindet sich demnach im Intervall
        $[1, n]$.

        Wir nehmen an, dass der Algorithmus $\COLOR(V, E, k)$ in
        deterministischer Polynomialzeit $p(n)$ entscheiden kann, ob der
        ungerichtete Graph $(V, E)$ mit $k$ Farben gefärbt werden kann.

        Folgender Algorithmus löst die Optimierungsvariante I von \COLOR{}:
        \begin{algorithmic}
          \REQUIRE{ein ungerichteter Graph $(V, E)$}
          \STATE{min $\leftarrow 1$}
          \STATE{max $\leftarrow |V|$}
          \WHILE{min $<$ max}
            \STATE{mid $\leftarrow \lfloor (\text{min} + \text{max})/2 \rfloor$}
            \IF{$\COLOR(V, E, \text{mid})$}
              \STATE{max $\leftarrow$ mid}
            \ELSE{}
              \STATE{min $\leftarrow$ mid}
            \ENDIF{}
          \ENDWHILE{}
          \RETURN{min}
        \end{algorithmic}

        Diese Binärschachtelung benötigt maximal
        $\lfloor \text{log}_2 \, n \rfloor + 1$
        Schritte.

        Der Graph wird auf dem Band einer Turingmaschine mit dem Bandalphabet
        $\{0, 1, \#\}$ wie folgt kodiert:
        \begin{align*}
          \underbrace{\bin(1) \# \bin(2) \# \ldots \# \bin(n)}_{\text{Knoten}}
          \# \#
          \underbrace{%
            \underbrace{\bin(1) \# \bin(2)}_{\text{Kante\ } (v_1, v_2)} \# \# \ldots
          }_{\text{Kanten}}
        \end{align*}
        Die Länge der binärkodierten Eingabe ergibt sich wie folgt:
        \begin{itemize}
          \item ein binärkodierter Knoten hat maximal die Länge
            $\bin(n) \leq \lfloor \text{log}_2\, n \rfloor + 1$
          \item davon haben wir $n$ Stück
          \item dazu kommen $n-1$ Trennzeichen zwischen den Knoten
          \item[$\Rightarrow$] maximal
            $n (\lfloor \text{log}_2\, n \rfloor + 1) + n-1
            \leq n (\lfloor \text{log}_2\, n \rfloor + 2)$
            Zeichen für die Knoten
          \item in einem Graphen existieren maximal
            $\binom{n}{2} = \frac{n (n-1)}{2}$
            Kanten
          \item eine binärkodierte Kante besteht aus $2$ Knoten, einem
            Trennzeichen zwischen den Knoten und zwei Trennzeichen vor dem
            ersten Knoten
          \item[$\Rightarrow$] maximal $\frac{n (n-1)}{2}
            \cdot \left(2
            \cdot (\lfloor \text{log}_2\, n \rfloor + 1) + 1 + 2\right)
            \leq n^2 (2 \lfloor \text{log}_2\, n \rfloor + 5)$
            Zeichen für die Kanten
        \end{itemize}
        Insgesamt liegt der Aufwand demnach in
        $O\Big(
          \big(
            n (\lfloor \text{log}_2\, n \rfloor + 2)
            + n^2 (2 \lfloor \text{log}_2\, n \rfloor + 5)
          \big) \cdot p(n)
        \Big)$, was die Polynomialzeiteigenschaft zeigt.
    \end{enumerate}

  \item
    \begin{enumerate}[(a)]
      \item Optimierungsvariante I für \TSP{}:

        Gegeben sei ein vollständiger Graph $(V, E = V \times V)$ mit
        einer Kostenfunktion $c: E \rightarrow \N$.

        Finde die minimalen Kosten einer Rundreise über alle Knoten.

      Optimierungsvariante II für \TSP{}:

        Gegeben sei ein vollständiger Graph $(V, E = V \times V)$ mit
        einer Kostenfunktion $c: E \rightarrow \N$.

        Finde die minimalen Kosten einer Rundreise über alle Knoten und einen
        Kreis über alle Knoten im Graphen, der diese Kosten erfüllt.

      \item
        \begin{enumerate}[1.]
          \item Wir nehmen an, dass der Algorithmus $\TSP(V, E, c, k)$ in
            deterministischer Polynomialzeit $q(n)$ entscheiden kann, ob in dem
            vollständigem Graphen $(V, E)$ mit der Kostenfunktion
            $c: E \rightarrow \N$ eine Rundreise über alle Knoten existiert,
            dessen Kosten $\leq k$ sind.

            Folgender Algorithmus löst die Optimierungsvariante I von \TSP{}:
            \begin{algorithmic}[1]
              \REQUIRE{ein vollständiger Graph $(V, E)$ und eine Kostenfunktion $c: E \rightarrow \N$}
              \STATE{lower $\leftarrow 0$}
              \STATE{$C \leftarrow \max\,\{c(e) \mid e \in E\}$}
              \STATE{upper $\leftarrow |V| \cdot C$}
              \WHILE{lower $<$ upper}
                \STATE{mid $\leftarrow \lfloor (\text{lower} + \text{upper})/2 \rfloor$}
                \IF{$\TSP(V, E, c, \text{mid})$}
                  \STATE{upper $\leftarrow$ mid}
                \ELSE{}
                  \STATE{lower $\leftarrow$ mid}
                \ENDIF{}
              \ENDWHILE{}
              \RETURN{lower}
            \end{algorithmic}

          \item Der Graph wird auf dem Band einer Turingmaschine mit dem
            Bandalphabet $\{0, 1, \#\}$ wie folgt kodiert:
            \begin{align*}
              \underbrace{\bin(1) \# \bin(2) \# \ldots \# \bin(n)}_{\text{Knoten}}
              \# \#
              \underbrace{%
                \underbrace{\bin(1) \# \bin(2)}_{\text{Kante\ } (v_1, v_2)}
                \#
                \underbrace{\bin(c((v_1, v_2)))}_{\text{Kosten von } (v_1, v_2)}
                \# \# \ldots
              }_{\text{Kanten}}
            \end{align*}
            Die Kodierung entspricht im Wesentlichen der von Aufgabe 2. (b), nur
            dass zu jeder Kante noch die Binärkodierung der Kostenfunktion
            hinzukommt.

            Sei $C$ der im Graphen größte angenommene Wert der Kostenfunktion.
            Ein Wert der Kostenfunktion nimmt dann maximal $\lfloor \text{log}_2
            C \rfloor + 1$ Stellen in Anspruch. Dazu kommt ein Trennzeichen pro
            Kante. Insgesamt kommen maximal $n^2 \cdot \left( \lfloor
            \text{log}_2 C \rfloor + 2 \right)$ Zeichen hinzu, sodass der
            Aufwand für die Optimierungsvariante I in
            \begin{align*}
              O\Big(
                \big(
                  n (\lfloor \text{log}_2\, n \rfloor + 2)
                  + n^2 (2 \lfloor \text{log}_2\, n \rfloor + 5)
                  + n^2 \left( \lfloor \text{log}_2 C \rfloor + 2 \right)
                \big) \cdot q(n)
              \Big)
            \end{align*}
            liegt, was die Polynomialzeiteigenschaft zeigt.
        \end{enumerate}

      \item
        \begin{enumerate}[1.]
          \item Wir nehmen an, dass der Algorithmus $\TSPOPTI(V, E, c)$ in
            deterministischer Polynomialzeit $r(n)$ in dem vollständigem Graphen
            $(V, E)$ mit der Kostenfunktion $c: E \rightarrow \N$ die minimalen
            Kosten einer Rundreise über alle Knoten berechnen kann.

            Folgender Algorithmus löst die Optimierungsvariante II von \TSP{}:
            \begin{algorithmic}[1]
              \REQUIRE{ein vollständiger Graph $(V, E)$ und eine Kostenfunktion $c: E \rightarrow \N$}
              \STATE{min $\leftarrow \TSPOPTI(V, E, c)$}
              \STATE{$U \leftarrow E$}
              \WHILE{$U \neq \emptyset$}
                \STATE{wähle $e$ aus $U$}
                \STATE{$U \leftarrow U \setminus \{ e \}$}
                \STATE{$k \leftarrow \TSPOPTI(V, E \setminus \{ e \}, c)$}
                \IF{$k =$ min}
                  \STATE{$E \leftarrow E \setminus \{ e \}$}
                \ENDIF{}
              \ENDWHILE{}
              \RETURN{$E$}
            \end{algorithmic}

          \item Dieser Algorithmus überprüft für jede Kante im Graphen, ob sie
            wichtig ist und zur Rundreise gehört oder ob die Rundreise ohne sie
            teurer würde. Am Ende bleiben nur die wichtigen Kanten übrig.

            Dabei wird jede Kante einmal überprüft (wovon maximal $n^2$
            existieren), sodass die Komplexität der Optimierungsvariante II in
            $O(n^2 \cdot r(n))$ liegt.
        \end{enumerate}
    \end{enumerate}
\end{enumerate}
