\begin{enumerate}[1.]
  \item
    \begin{enumerate}
      \item A1:
        \begin{align*}
          \begin{matrix}
              & a      & b      & c      & d      & e      & f\\
            a & \times &        & \times & \times & \times & \times\\
            b &        & \times & \times & \times & \times & \times\\
            c & \times & \times & \times &        & \times & \times\\
            d & \times & \times &        & \times & \times & \times\\
            e & \times & \times & \times & \times & \times & \\
            f & \times & \times & \times & \times &        & \times
          \end{matrix}
        \end{align*}
        \begin{align*}
          [a] & = \{ a, b \}\\
          [b] & = \{ a, b \}\\
          [c] & = \{ c, d \}\\
          [d] & = \{ c, d \}\\
          [e] & = \{ e, f \}\\
          [f] & = \{ e, f \}
        \end{align*}

      \item Der Myhill-Nerode-Index beträgt $3$, da A1 genau $3$
        Äquivalenzklassen besitzt ($\{a,b\}, \{c,d\}, \{e,f\}$).

      \item \
        \begin{center}
          \begin{tikzpicture}[scale=1,node distance=0.8cm,auto]
            \node [state, initial] (ab) {$a,b$};
            \node [state, accepting] (cd) [below=of ab] {$c,d$};
            \node [state] (ef) [below=of cd] {$e,f$};

            \path[->] (ab)  edge [loop right] node {$0$}    ();
            \path[->] (ab)  edge              node {$1$}    (cd)
                      (cd)  edge              node {$0,1$}  (ef);
            \path[->] (ef)  edge [loop right] node {$0,1$}  ();
          \end{tikzpicture}
        \end{center}

      \item A2:
        \begin{align*}
          \begin{matrix}
                 & z_0    & z_1    & z_2    & z_3    & z_4    & z_5    & z_6    & z_7\\
            z_0  & \times & \times & \times & \times &        &        & \times & \times\\
            z_1  & \times & \times & \times & \times & \times & \times & \times & \times\\
            z_2  & \times & \times & \times & \times & \times & \times & \times & \\
            z_3  & \times & \times & \times & \times & \times & \times & \times & \times\\
            z_4  &        & \times & \times & \times & \times &        & \times & \times\\
            z_5  &        & \times & \times & \times &        & \times & \times & \times\\
            z_6  & \times & \times & \times & \times & \times & \times & \times & \times\\
            z_7  & \times & \times &        & \times & \times & \times & \times & \times\\
          \end{matrix}
        \end{align*}
        \begin{align*}
          [z_0] & = \{z_0, z_4, z_5\}\\
          [z_1] & = \{z_1\}\\
          [z_2] & = \{z_2, z_7\}\\
          [z_3] & = \{z_3\}\\
          [z_4] & = \{z_0, z_4, z_5\}\\
          [z_5] & = \{z_0, z_4, z_5\}\\
          [z_6] & = \{z_6\}\\
          [z_7] & = \{z_2, z_7\}
        \end{align*}

      \item A2 hat den Myhill-Nerode-Index $5$, weil es genau $5$
        Äquivalenzklassen gibt:
        \begin{align*}
          \{z_0, z_4, z_5\}, \{z_1\}, \{z_2, z_7\}, \{z_3\}, \{z_6\}
        \end{align*}

      \item \
        \begin{center}
        \begin{tikzpicture}[scale=1,node distance=0.8cm,auto]
          \node [state, initial] (045) {$z_0, z_4, z_5$};
          \node [state] (1) [below=of 045] {$z_1$};
          \node [state, accepting] (27) [below=of 1] {$z_2, z_7$};
          \node [state] (3) [below right=of 27] {$z_3$};
          \node [state] (6) [below left=of 27] {$z_6$};

          \path[->] (045) edge [loop right] node        {$1$}   ();
          \path[->] (045) edge [bend left]  node        {$0$}   (1);
          \path[->] (1)   edge [bend left]  node        {$0$}   (045)
                          edge              node        {$1$}   (27);
          \path[->] (27)  edge              node        {$0$}   (3)
                          edge [bend left]  node        {$1$}   (6);
          \path[->] (6)   edge [bend left]  node        {$1$}   (27)
                          edge [bend right] node [swap] {$0$}   (3);
          \path[->] (3)   edge [loop right] node        {$0,1$} ();
        \end{tikzpicture}
        \end{center}

      \item Ein DEA ohne totale Zustandsübergangsfunktion hat Zustände, für die
        der Übergang bei bestimmten Zeichen nicht definiert ist. Das bedeutet,
        dass keine Zustandsänderung stattfindet und der DEA in diesem Zustand
        bleibt.

        Das heißt, dass man einen DEA ohne totale Zustandsübergangsfunktion zu
        einen mit totaler Zustandsübergangsfunktion erweitern kann, indem man
        zusätzliche Schleifen hinzufügt.

        Diesen DEA kann man dann ganz normal minimieren.
    \end{enumerate}

  \item
    \begin{enumerate}
      \item \
        \begin{center}
        \begin{tikzpicture}[scale=1, node distance=1cm, auto]
          \node [state, initial] (q_0) {$q_0$};
          \node [state] (q_1) [below right=of q_0] {$q_1$};
          \node [state] (q_3) [below left=of q_0] {$q_3$};
          \node [state, accepting] (q_2) [below left=of q_1] {$q_2$};

          \path[->] (q_0) edge node {$a$} (q_1)
                          edge node [swap] {$b$} (q_3)
                    (q_1) edge node {$a$} (q_3)
                          edge node {$b$} (q_2)
                    (q_3) edge [loop left] node {$a,b$} ()
                    (q_2) edge [loop below] node {$a,b$} ();
        \end{tikzpicture}
        \end{center}
      \item $L(A) = \{abw \mid w \in \{a, b\}^*\}$

      \item Da sich der Automat nicht weiter minimieren lässt, ist die minimale
        Pumpingzahl der Sprache gleich der Anzahl der Zustände des Automats und
        damit $4$.
      
    \end{enumerate}

\end{enumerate}
