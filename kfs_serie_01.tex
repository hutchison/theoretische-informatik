\begin{enumerate}
  \item
    \begin{enumerate}
      \item Die TM überprüft die \textsc{Palindrom}-Eigenschaft rekursiv, indem
        es das erste und das letzte Zeichen auf Gleichheit überprüft. Bei
        Gleichheit werden das erste und letzte Zeichen bei diesem Vorgang
        entfernt und der Vorgang beginnt erneut.

        Wörter der Länge $< 2$ sind dabei immer Palindrome.

      \item Wir betrachten das Wort $abaaba$ der Länge $n = 6$.

        Der Algorithmus der oben definierten TM arbeiten in vier „Phasen“, die
        jeweils unterschiedlich viele Schritte (Kopfbewegungen) benötigen.
        Wir nehmen an, dass der Kopf der TM zu Beginn am Anfang des Wortes
        steht.\footnote{Wir können die TM auch so verändern, dass der Kopf zu
        Beginn am Ende des Wortes oder sogar dahinter steht. Dazu müssen wir
        einfach eine zusätzliche Phase einbauen, die den Kopf zum Anfang des
        Wortes laufen lässt. Dies würde linearen Aufwand hinzufügen, was am Ende
        keinen Unterschied macht.}

        \begin{tabular}{cll}
          Phase & Beschreibung & benötigte Schritte\\
          \midrule
          I & erstes Zeichen lesen & $1$\\
          II & zum letzten Zeichen gehen & $6 = n$\\
          III & letztes Zeichen lesen & $1$\\
          IV & zum ersten Zeichen gehen & $5 = n-1$\\
          \midrule
          & & $13 = 2n+1$\\
          \bottomrule
        \end{tabular}

        Es bleibt das Wort $baab$ der Länge $4$ übrig. Der Algorithmus beginnt
        erneut und benötigt $2(n-2) + 1 = 9$ Schritte. Für das Wort $aa$ der
        Länge $2$ werden $2(n-4) + 1 = 5$ Schritte und für das leere Wort wird
        $1$ Schritt benötigt („springe in den akzeptierenden Zustand“).

        Um die Gleichheit des ersten und letzten Buchstabens eines Wortes der
        Länge $n$ zu überprüfen, werden $2n+1$ Schritte benötigt. Dies
        funktioniert auch für ein Wort der Länge $1$:

        \begin{tabular}{cll}
          Phase & Beschreibung & benötigte Schritte\\
          \midrule
          I & erstes Zeichen lesen & $1$\\
          II & zum letzten Zeichen gehen & $1$\\
          III' & in den akzeptierenden Zustand springen & $1$\\
          \midrule
          & & $3$\\
          \bottomrule
        \end{tabular}

        Die Anzahl der insgesamt benötigten Schritte berechnet sich durch
        folgende Summenformel:
        \begin{align*}
          \sum\limits_{k=0}^{\lfloor n/2 \rfloor} 2(n - 2k) + 1
        \end{align*}

        Für ein Wort der Länge $6$ werden insgesamt $13 + 9 + 5 + 1 = 28$
        Schritte benötigt.

        Die obere Grenze für den Index wird abgerundet, damit die Summation auch
        für Wörter ungerader Länge funktioniert.

        Für die genaue Abschätzung betrachten wir zunächst die Anzahl der
        Einsen. Diese beträgt $\lfloor n/2 \rfloor + 1$, da der Index bei $0$
        startet. Somit erhalten wir:
        \begin{align*}
          \sum\limits_{k=0}^{\lfloor n/2 \rfloor} 2(n - 2k) + 1
          = \lfloor n/2 \rfloor + 1 + \sum\limits_{k=0}^{\lfloor n/2 \rfloor} 2(n - 2k)
        \end{align*}

        Den konstanten Faktor $2$ können wir ohne Probleme herausziehen:
        \begin{align*}
          \lfloor n/2 \rfloor + 1 + \sum\limits_{k=0}^{\lfloor n/2 \rfloor} 2(n - 2k)
          = \lfloor n/2 \rfloor + 1 + 2 \sum\limits_{k=0}^{\lfloor n/2 \rfloor} n - 2k
        \end{align*}

        Den Term $\sum\limits_{k=0}^{\lfloor n/2 \rfloor} n - 2k$ können wir in
        zwei Summen aufspalten:
        \begin{align*}
          \lfloor n/2 \rfloor + 1 + 2 \sum\limits_{k=0}^{\lfloor n/2 \rfloor} n - 2k
          = \lfloor n/2 \rfloor + 1 + 2
            \left(
              \sum\limits_{k=0}^{\lfloor n/2 \rfloor} n
              - 2 \sum\limits_{k=0}^{\lfloor n/2 \rfloor} k
            \right)
        \end{align*}

        Der Term
        $\sum\limits_{k=0}^{\lfloor n/2 \rfloor} n$
        vereinfacht sich zu dem Produkt
        $\left( \lfloor n/2 \rfloor + 1 \right) n$
        und der Term
        $\sum\limits_{k=0}^{\lfloor n/2 \rfloor} k$
        ist die Gaußsche Summenformel, also
        $\frac{\lfloor n/2 \rfloor \left(\lfloor n/2 \rfloor + 1\right)}{2}$.

        Damit landen wir bei
        \begin{align*}
          \lfloor n/2 \rfloor + 1 + 2
            \left(
              \sum\limits_{k=0}^{\lfloor n/2 \rfloor} n
              - 2 \sum\limits_{k=0}^{\lfloor n/2 \rfloor} k
            \right)
          & = \lfloor n/2 \rfloor + 1 + 2
            \left(
              \left( \lfloor n/2 \rfloor + 1 \right) n
              - 2
              \frac{\lfloor n/2 \rfloor \left(\lfloor n/2 \rfloor + 1\right)}{2}
            \right)\\
          & = \lfloor n/2 \rfloor + 1 + 2
            \left( \lfloor n/2 \rfloor + 1 \right)
            \left( n - \lfloor n/2 \rfloor \right)
        \end{align*}

        Wir wählen für die weitere Analyse die Abschätzung
        $\lfloor n/2 \rfloor \leq \frac{n}{2}$
        und erhalten:
        \begin{align*}
          \lfloor n/2 \rfloor + 1 + 2
            \left( \lfloor n/2 \rfloor + 1 \right)
            \left( n - \lfloor n/2 \rfloor \right)
          & \leq \frac{n}{2} + 1 + 2
            \left( \frac{n}{2} + 1 \right)
            \left( n - \frac{n}{2} \right)\\
          & = \frac{1}{2} n^2 + \frac{3}{2} n + 1
            = \frac{1}{2} (n+2) (n+1)
            = \binom{n+2}{2}
        \end{align*}
        Eine kombinatorische Begründung fällt mir dafür aber nicht ein.

      \item --- % Es gilt $\frac{1}{2} n^2 + \frac{3}{2} n + 1 \in O(n^2)$.
        % Ist wohl falsch: „Laut VL: worst case des besten Programmes (??) O(n)“
        % Dabei kann ich (??) nicht entziffern.

      \item ---
    \end{enumerate}

    \item \begin{enumerate}
      \item Ist korrekt. Es ist zu zeigen, dass für fast alle $n$ und ein
        beliebiges $k \in \N \setminus \{ 0 \}$ gilt:
        \begin{align*}
          3n + 4 + \log n \leq k \cdot n
        \end{align*}
        Wir wählen $k := 5$ und erhalten:
        \begin{align*}
          3n + 4 + \log n & \leq 5n\\
          4 + \log n & \leq 2n = n + n
        \end{align*}
        Es gilt $4 \leq n$ für fast alle $n$ (nämlich ab $n > 4$).

        Bleibt zu zeigen, dass $\log n \leq n$.
        Wir wenden die $e$-Funktion auf beiden Seiten an und erhalten:
        \begin{align*}
          n \leq e^n = \sum\limits_{i=0}^{\infty} \frac{n^i}{i!} = 1 + n + \dots
        \end{align*}
        Daraus folgt $n \leq 1 + n$ für alle $n \in \N$. \hfill $\blacksquare$

      \item Ist korrekt. Es ist zu zeigen, dass für fast alle $n$ und ein
        beliebiges $k \in \N \setminus \{ 0 \}$ gilt:
        \begin{align*}
          \log n \leq k \cdot \sqrt{n}
        \end{align*}
        Wir wählen $k := 2$ und erhalten:
        \begin{align*}
          \log n & \leq 2 \sqrt{n}\\
          n & \leq e^{2 \sqrt{n}} \geq 1 + 2 \sqrt{n} + \frac{2 n}{2}\\
          n & \leq 1 + 2 \sqrt{n} + n\\
          0 & \leq 1 + 2 \sqrt{n}
        \end{align*}
        Die letzte Aussage ist für alle $n \in \N$ wahr. \hfill $\blacksquare$

      \item Ist nicht korrekt. Sei $k \in \N \setminus \{ 0 \}$ beliebig, aber
        fest gewählt.

        Es gilt
        \begin{align*}
          \sum\limits_{i = 1}^{n} i = \frac{n(n+1)}{2}
        \end{align*}
        Es wäre zu zeigen, dass für fast alle $n \in \N$ gilt:
        \begin{align*}
          \frac{n(n+1)}{2} & \leq k \cdot n\\
          n(n+1) & \leq 2k \cdot n
        \end{align*}
        Für $n > 0$ müsste gelten:
        \begin{align*}
          n+1 & \leq 2k
        \end{align*}
        Ab $n \geq 2k$ ist die Ungleichung aber nicht mehr erfüllt. Also gilt
        \begin{align*}
          \sum\limits_{i = 1}^{n} i \notin O(n)
        \end{align*}
        \ \hfill $\blacksquare$
    \end{enumerate}

    \item ---

    \item Die Komplexität des Suchalgorithmus „binäre Suche“ liegt in
      $O(\log n)$ bzw.\ genauer gesagt in $O(\log_2 n)$, da das Intervall, in
      welchem das Schlüsselement gesucht wird, bei jedem rekursiven Aufruf
      mindestens um die Hälfte verringert wird.

      Man könnte den Verlauf der binären Suche als Binärzahl interpretieren.
      Jede Stelle gibt entweder „suche links weiter“ ($0$) oder „suche rechts
      weiter“ ($1$) an. Im schlimmsten Fall wird immer nur links oder rechts
      gesucht. Falls immer nur rechts weitergesucht wird, erhalten wir die
      Binärdarstellung von $N-1$ und diese enthält $\log_2 (N-1)$ Einsen. Falls
      immer nur links weitergesucht wird, erhalten wir $\log_2 (N-1)$ Nullen.
\end{enumerate}
