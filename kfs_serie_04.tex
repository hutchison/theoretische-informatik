\begin{enumerate}
  \item \begin{enumerate}
    \item Wir zeigen: die Relation $\p$ ist auf der Menge \NP{} ist eine
      Quasiordnung.

      Seien
      $L_1 \subseteq \Sigma_1^*,
        L_2 \subseteq \Sigma_2^*,
        L_3 \subseteq \Sigma_3^*$
      Entscheidungsprobleme und
      $L_1, L_2, L_3 \in \NP$.

      \begin{description}
        \item[reflexiv] Es gilt $L_1 \p L_1$ genau dann, wenn eine polynomiale
          berechenbare Funktion $f: \Sigma_1^* \rightarrow \Sigma_1^*$
          existiert, so dass für alle Wörter $w \in \Sigma_1^*$ die Äquivalenz
          $w \in L_1 \Leftrightarrow f(w) \in L_1$ gilt.

          Diese Äquivalenz ist für $f := \id$ erfüllt, also gilt $L_1 \p L_1$.

        \item[transitiv] Es ist zu zeigen: $L_1 \p L_2$ und $L_2 \p L_3$
          impliziert $L_1 \p L_3$.

          $L_1 \p L_2$ bedeutet, dass eine polynomiale berechenbare Funktion
          $f: \Sigma_1^* \rightarrow \Sigma_2^*$ existiert, so dass für alle
          Wörter
          $w \in \Sigma_1^*$ die Äquivalenz
          $w \in L_1 \Leftrightarrow f(w) \in L_2$ gilt.

          Analog dazu bedeutet $L_2 \p L_3$, dass eine polynomiale berechenbare
          Funktion
          $g: \Sigma_2^* \rightarrow \Sigma_3^*$
          existiert, so dass für alle Wörter
          $w \in \Sigma_2^*$ die Äquivalenz
          $w \in L_2 \Leftrightarrow g(w) \in L_3$ gilt.

          Damit $L_1 \p L_3$ gilt, muss eine polynomiale berechenbare Funktion
          $h: \Sigma_1^* \rightarrow \Sigma_3^*$
          existieren, so dass für alle Wörter
          $w \in \Sigma_1^*$ die Äquivalenz
          $w \in L_1 \Leftrightarrow h(w) \in L_3$ gilt.

          Wir wählen $h := g \circ f$ und erhalten
          \begin{align*}
            w \in \Sigma_1^* & \Leftrightarrow f(w) \in \Sigma_2^*\\
            f(w) \in \Sigma_2^* & \Leftrightarrow g(f(w)) \in \Sigma_3^*
              & \text{also}\\
            w \in \Sigma_1^* & \Leftrightarrow h(w) = g(f(w)) \in \Sigma_3^*
          \end{align*}
          Da $f$ und $g$ jeweils polynomiale berechenbare Funktionen sind, ist
          auch $g \circ f$ eine polynomiale berechenbare Funktion und die
          Transitivität wurde gezeigt.
      \end{description}

    \item $\p$ ist auf $\NP$ nicht antisymmetrisch.

      In der Vorlesung wurde gezeigt, dass \CLIQUE{} \NP-vollständig ist, indem
      \DreiSAT{} auf \CLIQUE{} reduziert wurde. Sicherlich existiert folgender
      Beweis: $\CLIQUE \p \DreiSAT$, da beide Probleme \NP-vollständig sind.

      Wäre $\p$ auf $\NP$ antisymmetrisch, dann wären \CLIQUE{} und \DreiSAT{}
      das gleiche Entscheidungsproblem. Dies können sie aber nicht sein, weil
      die zugrundeliegenden Sprachen (einerseits Graphen und natürliche Zahlen
      und andererseits die Sprache der Aussagenlogik) nicht gleich sind.

    \item Die Menge der \NP-vollständigen Probleme ist auf der Relation $\p$
      eine Äquivalenzrelation.
  \end{enumerate}

  \item
    \begin{enumerate}
      \item Die vorgegebene Route vom Bahnhof zum Stadion sei der gegebene
        Graph, wobei die Straßenkreuzungen die Knoten und die Straßen zwischen
        den Kreuzungen die Kanten sind.

        Mit Hilfe der Optimierungsvariante 1 von $\IS$ können wir eine minimale
        unabhängige Menge von Knoten finden, so dass alle Knoten durch diese
        Menge verbunden sind. Mit der minimalen Menge unabhängiger Knoten wird
        sichergestellt, dass das Verhältnis
        \begin{align*}
          \frac{\text{Anzahl verfügbarer Polizisten}}{\text{Anzahl zu
          überwachender Straßen}}
        \end{align*}
        maximal wird.

      \pagebreak

      \item $\TSP$ beschreibt das folgende Problem:
        \begin{description}
          \item[Eingabe:] Ein ungerichteter Graph $G$ mit ganzzahliger
            Gewichtung der Kanten und einer Grenze $k \in \N$.
          \item[Ausgabe:] „Ja“ genau dann, wenn $G$ einen Hamiltonkreis enthält,
            dass die Summe der Kantengewichte kleiner oder gleich $k$ ist.
        \end{description}

        $\HC$ beschreibt das folgende Problem:
        \begin{description}
          \item[Eingabe:] ein ungerichteter Graph $G$
          \item[Ausgabe:] „Ja“ genau dann, wenn $G$ einen Hamiltonkreis enthält.
        \end{description}

        Wir zeigen $\HC \p \TSP$.

        Zunächst müssen wir $\TSP \in \NP$ zeigen. Dazu „raten“ wir einen Pfad
        in $G$ und überprüfen, ob dieser ein Hamiltonkreis ist und ob die Summe
        der Kantengewichte kleiner oder gleich $k$ ist. Dies ist in
        polynomieller Zeit möglich.

        Sei $G$ ein ungerichteter und ungewichteter Graph, bei dem zu
        entscheiden ist, ob er einen Hamiltonkreis enthält und sei $n$ die
        Anzahl der Knoten von $G$. Wir konstruieren einen gewichteten Graphen
        $G'$, dessen Knoten und Kanten mit denen von $G$ übereinstimmen, wobei
        jede Kante ein Gewicht von $1$ besitzt. Dies ist in polynomieller Zeit
        möglich. Gegeben sei außerdem der Grenzwert $k$, der mit der Anzahl der
        Knoten $n$ in $G$ übereinstimmt.

        Dann existiert ein Hamiltonkreis mit Gewichtssumme $n$ in $G'$ genau
        dann, wenn $G$ einen Hamiltonkreis enthält.
  \end{enumerate}
\end{enumerate}
